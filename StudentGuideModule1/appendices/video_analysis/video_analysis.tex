
\section{Video Analysis Using VideoPoint}

\textbf{Making a Movie} 

To make a movie, perform the following steps:

\begin{enumerate}
\item Start up {\it Videopoint Capture} by
going to {\tt Start} $\rightarrow$ {\tt Programs} $\rightarrow$
{\tt Physics Applications} $\rightarrow$ {\tt VideoPoint}
$\rightarrow$ {\tt VP Capture}.
\item The program will first ask you to choose a file name and location for 
the video you are going to make.  You should choose to put the file on the 
\textbf{Desktop} (so the data transfer will be fast enough).

\item Click on the {\tt Capture rate}
box and set the capture rate to 30 frames per second.
\item Go to the {\tt Size \& Colors} under the {\tt Capture Options} 
menu, and choose the largest available size for the video.

%\item Go to {\tt Preferences} under the {\tt Edit} menu.  Check the box
%that says {\tt Convert Captured File Before Editing.}  A dialog box will
%pop up; select the {\tt Intel Indeo Video 4.5} option and
%click OK.  Click OK \textbf{again} to close the Preferences box.
%(\textbf{If you forget to do this step, you won't be able to analyze
%the video.})

\item Be sure the camera is at least 2 meters from the object
you will be viewing. This constraint is required to reduce the effect
of perspective for objects viewed near the edge of the field of view.
Set up the camera so that its field of view is centered on the expected
region where you will perform the experiment. 
When you perform the experiment only use the data when the object is in this central region.

\item Focus the camera by rotating the barrel on the outside
of the lens until you have a clear picture.

\item Place a meter stick or some
object of known size in the field of view where it won't interfere
with the experiment. The meter stick should be the same distance away from
the camera as the motion you are analyzing so the horizontal and vertical 
scales will be accurately determined. It should also be parallel to one of the
sides of the movie frame.

\item One member of your group should perform the computer tasks while the
other does the experiment.

\item To start recording your video, click {\tt Record.}  When
you're done, click {\tt Stop.}

\item The next step is to decide how much of the movie to save.
Use the slider to step through the movie frame by frame.  When you
find the first frame you want to save, click {\tt First.}  When you find
the last frame you want to save, click {\tt Last.}  
You may want to save the 
entire movie, in which case {\tt First} and {\tt Last} really will be the
first and last frames.  Often, though, there will be ``dead'' time
either at the beginning or the end of the movie, which you might
as well cut out before saving.

\item After you've selected the range of frames you want to save, the button
at the lower right should say {\tt Keep}.  Make sure that the 
box next to this button says {\tt All}.  (If it says {\tt Double},
change it to {\tt All}.)
Then click {\tt Keep}
and {\tt Save.}  You will be asked to provide a file name. Pick something unique that you can easily identify.
You will then see a quick replay of the movie as Videopoint
converts and saves it.

%\item Click {\tt Open in Video Point.}

\end{enumerate}
\textbf{Analyzing the Movie} 

To determine the position of an object at different times during the
motion, perform the following steps:

\begin{enumerate}

\item Start up {\it Videopoint} by
going to {\tt Start} $\rightarrow$ {\tt Programs} $\rightarrow$
{\tt Physics Applications} $\rightarrow$ {\tt VideoPoint}
$\rightarrow$ {\tt VideoPoint 2.5}.
Click {\tt Open Movie}. You will see a dialog box. Set the {\tt Files of type:}
box to {\tt All Files(*.*)} and select the file you created before.
The file should have an extension like `.mov' or `.movvv'.
Click {\tt Open}.

\item {\tt VideoPoint} will request the number of objects you want to
track in the movie. Carefully read the instructions for the unit you
are working on to find this number. Enter it in the space provided.
You will now see several windows. 
(Note: You may have to move the movie window out of the way to see the
other windows.) One contains the movie and has control
buttons and a slider along the bottom of the frame to control the
motion of the film. Experiment with these controls to learn their
function. Another window below the movie frame (labeled Table) contains position 
and time data and a third window to the right of the frame (labeled Coordinate
Systems) describes the coordinate system in use.

\item This is a good time to calibrate the scale. Go to a frame where an
object of known size is clearly visible (see item 7
in the previous
section). Under the \textbf{Movie} menu highlight \textbf{Scale Movie}.
A dialog box will appear. Enter the length of the object and set \textbf{Scale
Type} to \textbf{Fixed}. Click \textbf{Continue}. Move the cursor
over the frame and click on the ends of scaling object.

\item You are now ready to record the position and time data. Go to the
first frame of interest. Move the cursor over the frame and it will
change into a small circle with an attached label. Place the circle
over the object of interest in the frame and click. The $x$ and $y$ positions
will be stored and the film advanced one frame. Move the circle over
the position of the object in the frame and repeat. Continue this
process until you have mapped out the motion of the object. If you
entered more than one object to keep track of when you opened the
movie, then you will click on all those objects in each frame before
the film advances.
Remember to restrict your analysis to the central region of the field of view to
reduce any distortion created by the camera lens.
For example, to study free-fall the ruler used for setting the scale should be in the
central region of the camera's field of view. You should only take data while the object is falling from one
end of the ruler to the other.

\item When you have entered all the points you want, go to the
\textbf{File} menu and select \textbf{Export data}.  This will
allow you to save your data table as an Excel file. 
Save this
file on the desktop (by clicking on the ``Open'' button, which actually
doesn't open anything), and double-click on the saved file to start up Excel.
You will now be able to continue your data analysis in Excel.

\item Once you have looked at your data in Excel and made sure everything
looks OK, you can quit Videopoint Analysis.  If you are sure you have
exported your data correctly to Excel, there is no need to save in Videopoint.

\end{enumerate}
\textbf{Changing the Origin} 

To change the position of the origin take the following steps.

\begin{enumerate}
\item Click on the arrow icon near the top of the menu bar to the left.
The cursor will be shaped like an arrow when you place it on the movie
frame. 
\item Click at the origin (where the axes cross) and drag the origin to
the desired location. 
\item Click on the circle at the top of the menu bar to the left to return
to the standard cursor for marking points on the film. 
\end{enumerate}
\textbf{Using a Moving Coordinate System} 

To record the position of an object and to change the coordinate system
from frame to frame take the following steps.

\begin{enumerate}
\item Open the movie as usual and enter one object to record. First we have
to select the existing origin and change it from a fixed one to a
moving one. Click on the arrow near the top of the menu bar to the
left. The cursor will have the shape of an arrow when you place it
on the movie frame. Click on the existing origin (where the axes cross)
and it will be highlighted.
\item Under the \textbf{Edit} menu drag down and highlight \textbf{Edit
Selected Series}. A dialog box will appear. Click on the box labeled
\textbf{Data Type} and highlight the selection \textbf{Frame-by-Frame}.
Click OK.
\item Click on the circle at the top of the menu bar to the left to change
the cursor back to the usual one for marking points. Go to the first
frame of interest. When the cursor is placed in the movie frame it
will be labeled with ``Point S1.'' Click on the object of interest.
The film will NOT advance and the label on the cursor will change
to ``Origin 1.'' Click on the desired location of the origin
in that frame. The film will advance as usual. Repeat the procedure
to accumulate the $x$- and $y$-positions relative to the origin you've
defined in each frame.\end{enumerate}

