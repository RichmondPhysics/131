
\section{Air Column Resonance\footnote{
Adapted from \underbar{Experiments in Physics}, 6th ed., B. Cioffari, D. Edmonds,
Jr. D. C. Heath and Co., 1978.
}}

\makelabheader %(Space for student name, etc., defined in master.tex or labmanual_formatting_commands.tex)



{\noindent \bf Objectives:} \begin{list}{$\bullet$}{\itemsep0pt \parsep0pt}

\item Measure the wavelengths in air of sound waves of different frequencies.

\item Determine the velocity of sound in air.

\end{list}

\noindent {\bf Introduction} 

The speed
of a wave is directly proportional to both the frequency of vibration and the length of the wave,

\begin{equation} v = \lambda f \end{equation}

\noindent A tuning fork held over the open end of a tube will send air disturbances (compressions and rarefactions) down the length of the tube. As the medium changes at the other end of the tube, the disturbances will be reflected. Should these reflections be in phase with the waves generated by the tuning fork, resonance occurs, that is, the disturbances reinforce one another and produce a louder sound.

\noindent In order that the tube resonate, the frequency of the vibrating air must coincide with the natural frequency of the tube (which may be its fundamental or one of its overtones). For a tube like the one used in this experiment, which is closed at the opposite end, this requirement is met if the tube length is an odd number of quarter wavelengths of the sound waves produced by the fork ($l = \lambda/4, 3 \lambda/4, 5 \lambda/4$, etc., where $l$ is the length of the tube and $\lambda$ is the wave length of the sound). Note that if the length of the tube is gradually increased while the fork is vibrating, the distance between successive resonance positions is $\lambda/2$. \\

\noindent {\small {\bf Note:} Due to edge effects at the open end of a tube, the length of the wave depends on the radius of the opening. Thus, $l_{eff} = l + 0.6r$, where $l_{eff}$ is the \textit{effective} length, $l$ is the length measured, and $r$ is the tube radius.} \\

\noindent By raising and lowering a brass tube inserted vertically into a larger plastic tube filled with water, column length may be varied. The (effective) length at resonance gives the wavelength; the tuning fork frequency gives the sound frequency. Formula 1 then gives the velocity. \\

{\noindent \bf Apparatus:} \begin{list}{$\bullet$}{\itemsep0pt \parsep0pt}

\item resonance tube setup \item 2 tuning forks of different pitches \item meter stick \item rubber hammer \item thermometer

\end{list}

{\noindent \bf Activity:} \begin{enumerate}

\item You will first determine the shortest tube length for resonance. Do this as follows: Start the tuning fork vibrating by striking it gently with the rubber hammer. Be sure the brass tube is in its lowest position and the fork extends over the open end in such a manner that the prongs vibrate vertically.

\item Slowly raise the brass tube until the first position of resonance is reached.

\item Measure and record the distance from the open end of the tube to the water level in the plastic tube.

\item Raise the tube until the next position of resonance is reached and measure and record the distance as before.

\item Continue with a third position of resonance.

\item Repeat 1- 5 with the same fork twice more for a total of three independent measurements of resonance positions.

\item Repeat 1- 6 with a second tuning fork of a different frequency.

\end{enumerate}

\vskip35pt

\noindent {\Large{\bf DATA}} \\

\noindent Room Temperature ($^\circ$C) \hrulefill \ \  Tube radius (m) \hrulefill

\begin{center} \begin{tabular}{||c|c|c|c|c|c|c|c|c|c|c|c|c|c|c||} \hline \hline Tuning Fork & \multicolumn{4}{|c|}{First Position of} & \multicolumn{4}{|c|}{Second Position} & \multicolumn{4}{|c|}{Third Position of} & Wave- & Velocity of \\ Frequency, & \multicolumn{4}{|c|}{Resonance, m} & \multicolumn{4}{|c|}{of Resonance, m} & \multicolumn{4}{|c|}{Resonance, m} & length, & Sound in \\ \cline{2-13} Hz & 1 & 2 & 3 & Ave. & 1 & 2 & 3 & Ave. & 1 & 2 & 3 & Ave. & m & air, m/s \\ \hline \hline &&&&&&&&&&&&&& \\ \hline &&&&&&&&&&&&&& \\ \hline \hline \end{tabular} \end{center}


{\noindent \bf Analysis:} \begin{enumerate}

\item For each (effective) length of resonating air column, calculate the average of the three readings.

\item For each tuning fork, determine the value of the wavelength of its sound wave in air from the average lengths of the resonating column.

\item For each tuning fork, use the tuning fork frequency and the wavelength determined to calculate the velocity of sound in air.

\item Average your two values to determine your experimental velocity of sound in m/s: \ \ \  \rule{2cm}{1pt}

\item The velocity of sound in air at $0^\circ$C is 331.4 m/s.  The temperature dependence of sound velocity in air is given by $v(T) = 331.4 + 0.6T$, where $T$ is in $^\circ$C and $v$ is in m/s. Calculate an ``accepted'' value of the velocity of sound in air from this formula.

\vskip35pt

\item What is the percent difference between your experimental result and the ``accepted'' value?

\end{enumerate}

\vskip35pt

\noindent{\bf Questions:} 

\begin{enumerate}
\item Suppose that, in this experiment, room temperature had been lower. How would
the length of the resonating air column have changed? Explain. \vspace{20mm}

\item How would an atmosphere of hydrogen affect the pitch of a tuning fork? Explain.\vspace{20mm}

\item A tuning fork rated at 128 vibrations per second is held over a resonance tube.
What are the two shortest distances at which resonance will occur at a temperature
of $20^\circ$C?\vspace{20mm}

\item The first resonance of a tube is found to occur when the position of the water
line below a vibrating tuning fork is at 10 cm. A second resonance is found
at 26 cm. What is the frequency of the tuning fork? (Take the air temperature
to be 20$^\circ$C.)
\end{enumerate}
