
\section{The Derivatives of the Sine and Cosine Function}

\makelabheader %(Space for student name, etc., defined in master.tex or labmanual_formatting_commands.tex)

\bigskip
\textbf{Objectives }

\begin{itemize}[nosep]
\item Develop mathematical tools to study the harmonic oscillator.
\item To apply an oft-used approximation to the sine and cosine functions function to calculate their derivatives.
\item To compare the previous results to analytical methods.
\end{itemize}

\medskip
\textbf{Overview }

Periodic motion is common in nature and the simple harmonic oscillator ({\it i.e.} a spring) is
a common example.
The motion of a system undergoing such motion is usually described with sine
and/or cosine functions like the following
\begin{equation}
x(t) = A\cos(\omega t + \phi)
\end{equation}
where $A$ is the amplitude of the oscillation, $\omega$ is the angular frequency, $\phi$ is the
phase angle, and $t$ is time.
Inserting this expression for $x(t)$ into Newton's Second Law for an oscillator
with spring constant $k$
\begin{equation}
F_s = -kx = ma = m\frac{d^2x}{dt^2}
\end{equation}
which shows we need to take derivatives of the sine and cosine to study the harmonic oscillator.

\textbf{Activity 1: The Derivative of the Sine} 

The definition of the derivative of a function $f(t)$ is 
\begin{equation}
\frac{d \, f \! (t)}{dt} = \lim_{\Delta t \rightarrow 0} \frac{f(t+\Delta t) - f(t)}{\Delta t}
\end{equation}
which can be approximated as
\begin{equation}
\frac{d \, f \! (t)}{dt} \approx \frac{f(t+\Delta t) - f(t)}{\Delta t}
\label{eq:approx}
\end{equation}
if $\Delta t$ is small enough.
We will use this approximation to calculate derivatives and compare the results
with the analytical form of the derivatives.

(a) First make a plot of the sine function using {\it Excel}. 
There are instructions in Appendix E of {\it Physics For Doing!}
Use the first column to hold the angles for the points along the curve.
Go to the top row of the first column and
enter `Angle' (cell A1).
Go to the cell below (cell A2) and enter `=(ROW()-2)*0.1' where the
`ROW()' function returns the value of the row for that cell. We subtract two from
the row number so the sine curve will start at zero, and then multiply by a
scale factor 0.1 which corresponds to a time step of $0.1~s$.

(b) Click in the A2 cell and then click and drag down on the small square in the lower, right corner
of the cell. Drag down to row 65 of so.
{\it Excel} will fill the cells with the angle data.
Enter `Sine' in cell B1 at the top of the second column.
Click in cell B2, enter `=SIN(A2)'.
Click and drag down on the square in the lower, right corner of cell B2 to the same bottom row 
you used before.

(c) Now make a scatterplot the data in columns A and B. Highlight both columns of data and 
go to {\tt Insert} and select scatterplot in the {\tt Charts} menu. You should see 
a plot of the sine.

\pagebreak 

(d) Now go to cell C1 and enter `dx/dt'. In cell C2 calculate the slope of the sine
function between the time for this row and the next one (row 3).
Click and drag the small square in the lower, right corner of
cell C2 to the second to last row of your angle calculation.
Why shouldn't you go all the way to the last row?

\vspace{1.8cm}

(e) Add the data in column C to your plot. Right-click in your plot and then left-click
on `Select data'. You should see the `Select Data Source' window. Click {\tt Add}.
You should now get the `Edit Series' window. Make the appropriate entries
for the series name, $x$ values, and $y$ values.
When you're done making the plot, click on the large plus sign at upper, right and check
the boxes for `Legend' and `Axis Titles'.

(f) What familiar function does the approximate derivative curve look like?
If you're having trouble with this, consult your instructor.

\vspace{1.8cm}

(g) Your curve for $dx/dt$ should closely resemble a cosine which is equal
to the derivative of the sine function.
To check the accuracy of your approximation add a cosine curve to your plot
using the same techniques described in steps (b) and (e). 
How does the cosine curve compare with your approximate derivative of the sine?

\vspace{1.8cm}

You should have found your approximate derivative of the sine function closely 
resembles the cosine.
This is the type of approximation routinely used in physics, for example, in the labs
where you measure velocity and acceleration extracted from position measurements.
Keep your spreadsheet and plot for Activity 2.
Print your plot and attach it to this unit.


\textbf{Activity 2: The Derivative of the Cosine} 

Since we will also need the derivative of the cosine function in our study of the
harmonic oscillator, we now apply the same approximate derivative formula (Eq. 4)
to the cosine.

(a) You should already have the data for a cosine curve in your spreadsheet.
Use that data and the same procedures used in Activities 1.(d)-(e) to calculate
the derivative of the cosine and add that data to your spreadsheet.

(b) Add the data for the approximate derivative of the cosine to your plot.
What familiar function does the approximate derivative curve resemble?
If you're having trouble with this, consult your instructor.

\vspace{1.8cm}

(c) How does the sine curve on your plot from Activity 1 compare with 
your approximate derivative of the cosine?

\vspace{1.8cm}

You should have found your approximate derivative of the cosine function closely 
resembles the negative of the sine.
We will use both of these derivatives demonstrated here frequently
when investigating the simple harmonic oscillator.
