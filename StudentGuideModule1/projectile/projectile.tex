
\section{Projectile Motion\footnote{
1990-93 Dept. of Physics and Astronomy, Dickinson College. Supported by FIPSE
(U.S. Dept. of Ed.) and NSF. Portions of this material have been modified locally
and may not have been classroom tested at Dickinson College.
}}

\makelabheader %(Space for student name, etc., defined in master.tex or labmanual_formatting_commands.tex)

\textbf{Objectives }

To understand the experimental and theoretical basis for describing projectile
motion as the superposition of two independent motions: (1) a body falling in
the vertical direction, and (2) a body moving in the horizontal direction with
no forces.

\textbf{Apparatus}

\begin{itemize}
\item A tennis ball. 
\item A movie scaling ruler or meter stick.
\item A video analysis system (\textit{Tracker}). 
\item Graphing and curve fitting software (\textit{Excel}).
\end{itemize}
\textbf{Activity 1: Predicting the Two-Dimensional Motion of a Tossed Ball }

(a) Toss a tennis ball up at an angle of about 60\( ^{\circ } \) with the horizontal
a couple of times. Sketch the motion and describe it in words below. What is
the shape of the trajectory?
\vspace{20mm}

(b) Let's consider the horizontal and vertical components of the motion separately.
What do you think is the horizontal motion of the ball? Is it motion with constant
velocity, constant acceleration, or some other kind of motion? (Hint: What
is the force acting on the ball in the horizontal direction after it is released?)
\vspace{20mm}

(c) What do you think is the vertical motion of the ball? Is it motion with
constant velocity, constant acceleration, or some other kind of motion?
(Hint: What is the force acting on the ball in the vertical direction after
it is released?)
\vspace{20mm}

The two-dimensional motion of a tossed ball is too fast to observe carefully
by eye without the aid of special instruments. In the next activity we will
use a video analysis system to study the motion of a small ball launched at
an angle of about 60\( ^{\circ } \) with respect to the horizontal. You are
to use the video analysis software and mathematical modeling techniques to find
the equations that describe: (a) the trajectory ($y$ \textit{vs.}~$x$), 
(b) the horizontal
motion ($x$ \textit{vs.}~$t$), and (c) the vertical motion ($y$ \textit{vs.}~$t$) of the projectile.

\newpage
\textbf{Activity 2: Analyzing Projectile Motion} 

(a) Make a movie of a tennis ball in flight by following these steps. 

\begin{enumerate}
\item Open \textbf{Camera} and turn on the video camera as explained in \textbf{Appendix \ref{tracker}: Video Analysis Using Tracker}. Center the field of view of the camera on the region where you will toss the ball. This region should be about 2 meters from the camera to
get a large enough area for the flight of the ball. Place a ruler or meter stick
somewhere in the field of view close to the plane of the motion of the ball
where it won't interfere with the motion. This ruler will be used later to determine the scale. 
\item Make a movie of the tennis ball flying through the air with a significant component
of its initial velocity in the horizontal direction (i.e., don't toss it straight
up). Make sure most of the complete trajectory is visible to the camera. See
\textbf{Appendix \ref{tracker}: Video Analysis Using Tracker} for details on making the movie. 
%When you
%save the movie file give it the name \textit{Projectile}.

\end{enumerate}
%(b) Determine the position of the projectile at different times during the 
%motion. If using \textbf{VideoPoint}, follow the instructions in the second section of 
%\textbf{Appendix \ref{videopoint}: Video Analysis} for recording and calibrating the position data. When you
%are analyzing the movie place the origin at the position of the ball in your
%first frame. Do this by clicking on the arrow symbol on the menu bar to the
%left of the movie frame. The cursor will take the shape of an arrow when you
%place it over the frame. Move the point of the cursor's arrow to the origin.
%Click and drag the cursor and move the origin to the position of the ball in
%the first frame and release. This sets the origin at the location of the ball
%at the initial time. When you have placed the origin at the desired spot, click
%on the circular icon at the top of the menu bar to the left of the movie frame.
%This returns the cursor to a circle that marks the position of the ball.
%When you have finished marking the ball's position, export the data into
%an \textit{Excel} file as described in \textbf{Appendix \ref{videopoint}}.
%
%(c) Open your data in \textit{Excel}.
%Launch \textit{Excel}. 
%See \textbf{Appendix \ref{excel}: Introduction to Excel} for details on using
%\textit{Excel}. Make a plot of the vertical position ($y$) versus the
%horizontal position ($x$). Determine the equation that describes the trajectory
%of the projectile by including a trendline using a polynomial fit of order 2. 
% When you have found a good fit to the data, print the graph
%and attach a copy to this unit. Write the equation for the trajectory of the
%projectile in the space below. Be sure to include the proper units. What is the shape of the trajectory? Does the result agree with your earlier prediction?
%\vspace{20mm}

(b) Determine the position of the ball as a function of time using \textit{Tracker}. To do this, 
track the position of the ball throughout its trajectory.  The resulting file will contain three columns 
with the values of time, $x$-position and $y$-position.

(c) Determine the equation that describes the trajectory of the projectile by plotting the vertical 
position $y$ versus the horizontal position $x$. Write the equation for the trajectory of the projectile in the space below. Be sure to include the proper units. What is the shape of the trajectory? Does the result agree with your earlier prediction?
\vspace{10mm}

(d) Determine the equation that describes the horizontal motion of the projectile by plotting the horizontal position ($x$) versus time ($t$) in \textit{Tracker}.
%\textit{Excel}, again including a trendline using a polynomial fit of order 2. 
%When you have
%found a good fit to the data, print the graph and attach a copy to this unit.
What kind of motion is it? What would you expect for the horizontal acceleration? Does the result agree with your earlier prediction?
\textbf{Note:} As in the previous experiment, numbers from your graph should be rounded off to no more than 3 significant figures.
\vspace{8mm}

\begin{enumerate}
\item The equation for the horizontal component of the motion with proper units is:
$x =$\vspace{5mm}

\item The horizontal component of the acceleration with proper sign and units is:
\( a_{x}= \) \vspace{5mm}

\item The horizontal component of the initial velocity with proper sign and units
is: \( v_{0x}= \)\vspace{5mm}

\item The initial $x$ position with proper units is: \( x_{0}= \)\vspace{5mm}

\end{enumerate}
(e) Determine the equation that describes the vertical motion of the projectile
by plotting the vertical position ($y$) versus time ($t$) in \textit{Tracker}. 
%again including a trendline using a polynomial fit of order 2.  When you have found a good fit to the data, 
Print the graph and attach a copy to this unit. 
What kind of motion is it? What would you expect for the vertical acceleration? Does the result agree with your earlier prediction?
\vspace{8mm}

\begin{enumerate}
\item The equation for the vertical component of the motion with proper units is:
$y =$\vspace{5mm}

\item The vertical component of the acceleration with proper sign and units is: 
\( a_{y}= \)
\vspace{5mm}

\item The vertical component of the initial velocity with proper sign and units is:
\( v_{0y}= \) \vspace{5mm}

\item The initial $y$ position with proper units is: \( y_{0} =\) \vspace{5mm}

\end{enumerate}
(f) Does it appear that projectile motion is simply the superposition of two
types of motion that we have already studied? Explain.
\vspace{20mm}

(g) Go around to the other groups in the lab and ask them for their measured values of the horizontal and vertical accelerations.
Make a histogram of your results for each component and calculate the average and standard deviation of each one.
For information on making histograms, see \textbf{Appendix \ref{excel}}. For information on calculating the average and
standard deviation, see \textbf{Appendix \ref{treatment}}. Record the average and standard deviation here.
Attach the histogram to the unit.
\vspace{30mm}

(h) What is your expectation for the vertical acceleration of the ball? Is your data consistent with your expectation?
Is the acceleration the same for the entire class? 
Use the average and standard deviation for the class to quantitatively answer these questions.
\vspace{25mm}

(i) What is your expectation for the horizontal acceleration of the ball?  Is your data consistent with your expectation?
Is this acceleration the same for the entire class? 
Use the average and standard deviation for the class to quantitatively answer these questions.
\vspace{25mm}

(j) What do the histograms of the class data for each component of the acceleration tell you? Be quantitative in your answer.
\vspace{20mm}

