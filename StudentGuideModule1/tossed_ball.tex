
\section{The Tossed Ball}

\makelabheader %(Space for student name, etc., defined in master.tex or labmanual_formatting_commands.tex)

{\noindent \bf Objectives:} \begin{list}{$\bullet$}{\itemsep0pt \parsep0pt}

\item Application of the free-fall concept \item Investigation of linear motion in two directions with constant acceleration

\end{list}

{\noindent \bf Introduction}

{\noindent In the Free Fall investigation, you determined the acceleration of objects allowed to fall from rest. In this lab you will study this sort of motion with an initial velocity.}

\medskip

{\noindent \bf Predictions}

{\noindent If you throw a ball straight up into the air, releasing it at the level of the table-top with enough velocity so that it just reaches the top of a two-meter stick resting on the floor, how long will the ball's flight take from its release until it hits the floor? List all your assumptions and the measurements you will make, and show all your calculations in making this prediction.}

\vspace{100pt}

{\noindent \bf Apparatus:} \begin{list}{$\bullet$}{\itemsep0pt \parsep0pt}

\item 2-meter stick \item tennis ball \item 2 stop watches

\end{list}

{\noindent \bf Activity:} \begin{enumerate}

\item To test your predictions, stand the two-meter stick on its end. One of the experimenters should be positioned to view the high end of the stick.

\item A second experimenter throws the ball up in such a way that the ball is released at the level of the table-top and just reaches the top of the two-meter stick. You might try grasping the table-edge with your non-throwing arm extended out over the floor. Throw from under this extended arm, releasing the ball as your two arms make contact. One or more stop watches should be started at the instant of release. Repeat as necessary until the maximum height of the ball is as close to two-meters as possible.

\item Time the flight of the ball from its release until it hits the floor. When you get a good toss, record the value and compare it to your prediction.

\end{enumerate}

\bigskip

{\noindent Time predicted \rule{1.75in}{0.2pt} \hspace{10pt} Time measured \rule{1.75in}{0.2pt}}

\pagebreak

{\noindent \bf Questions:}

\begin{enumerate}
\item You throw a ball straight up in the air. Some time later, the ball falls to
the ground. Is the acceleration constant during the entire flight of the ball?
Explain.\vspace{20mm}

\item What is (are) the value(s) and direction(s) of the acceleration at the instant
the ball is released? At its highest point? On the way down? Just before it
hits the floor?\vspace{20mm}

\item Does your measurement agree with your prediction? Explain any discrepancy.\vspace{20mm}

\item What was the initial velocity of the ball?\vspace{20mm}

\item What was the ball's velocity when it reached the level of the table-top on its
way down?\vspace{20mm}

\item What was the ball's velocity when it hit the ground?
\end{enumerate}
