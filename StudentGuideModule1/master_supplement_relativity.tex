\documentclass[english,twoside]{article}

%Various Latex formatting parts are here, to keep master.tex files relatively clean.

%---------------------------------------------------------------------------------------------------
%Things related to fonts and overall page appearance

\usepackage{lmodern}
\usepackage{babel}
\usepackage[T1]{fontenc}
\usepackage[nomarginpar]{geometry}
\geometry{verbose,letterpaper}
\addtolength{\oddsidemargin}{1.0cm} %without these two lines, larger margin is on the OUTSIDE.  We want the larger edge on the INSIDE, to allow room for the three hole punches
\addtolength{\evensidemargin}{-1.0cm}
\setlength\topmargin{0.2in}
\addtolength{\hoffset}{-1.0cm}
\addtolength{\textwidth}{2.0cm}
\addtolength{\voffset}{-1.5cm} %This line is apparently needed on some versions of MikTex XeLatex.  Comment out if your pages appear shifted too high.
\addtolength{\textheight}{3.5cm}

\usepackage{fancyhdr}
\pagestyle{fancy}
%\fancyhead[LE,RO]{\slshape \rightmark} %This is the default for fancy page style
%\fancyhead[LO,RE]{\slshape \leftmark}
\fancyhead[LO,RE]{\slshape \rightmark} 
\fancyhead[LE,RO]{\slshape \leftmark} % Reversed LE, RO to  LO,RE to make headers come out correctly on even/odd pages

\usepackage{tocloft} %Allow us to leave page numbers for Parts out of table of contents
\cftpagenumbersoff{part} %No page numbers for Parts out of table of contents
\renewcommand{\cftsecdotsep}{\cftsubsecdotsep}
\setlength\parskip{\medskipamount}
\setlength\parindent{0pt}

\usepackage{xcolor}
\usepackage[pagecolor={white},nopagecolor={white}]{pagecolor}
\usepackage{microtype} % (apparently not compatible with pdflatex)
%\usepackage{epstopdf} %this package apparently allows pdflatex to work on this document, since all we use are eps figures.

%---------------------------------------------------------------------------------------------------
%Footnotes
% Note by MT, 3/19/2017: I have no idea what the code below is actually doing.  
% But the Latex compiler crashes whenI take it out, so for now I'm leaving it in.

%% Special footnote code from the package 'stblftnt.sty'
%% Author: Robin Fairbairns -- Last revised Dec 13 1996
\makeatletter
\let\SF@@footnote\footnote
\def\footnote{\ifx\protect\@typeset@protect
    \expandafter\SF@@footnote
  \else
    \expandafter\SF@gobble@opt
  \fi
}
\expandafter\def\csname SF@gobble@opt \endcsname{\@ifnextchar[%]
  \SF@gobble@twobracket
  \@gobble
}
\edef\SF@gobble@opt{\noexpand\protect
  \expandafter\noexpand\csname SF@gobble@opt \endcsname}
\def\SF@gobble@twobracket[#1]#2{}
\makeatother


%---------------------------------------------------------------------------------------------------
%Things related to sectioning and including files

\usepackage{titlesec}
\usepackage{xr} %used to allow external references, so that master_supplement.tex can reference appendices, for instance.
%\usepackage{newclude} %Allows use of /include*{}
%DANGER DANGER: newclude is NOT compatible with package xr, used for external references.

\usepackage{import}
\usepackage{xparse}
% The following lines define a new command \includelab, for including labs in the master.tex file.
% Usage is \includelab{1}{file} to include it, or \includelab{0}{file} to NOT include it.  
% But all 0's can be overridden by writing \includealllabstrue in the master.tex file, which is easier than deleting 
% fifty individual `%' signs and then remembering to put them all back, which is what you had to do before.
% \includeonly still works as you expect it to.
%
% You can also specify a directory for the file, as in \includelab{1}[path]{file}.
% This provides a clean way to share files between 131, 132, and 125.  For example, within the 132 master file,
% if you want to call a 131 lab, you would use: 
%        \includelab{1}[../../131/StudentGuideModule1/]{velocity/velocity}.  
% Among other benefits, this creates a file velocity/velocity.aux within the 132 directory structure, 
% so that the aux file from 132 does not interfere with the aux file from 131.
\newif\ifincludealllabs

\NewDocumentCommand \includelab { m o m } {%
  \IfNoValueTF  {#2}  {%NO directory {#1} {#2} {#3} \par 
	\ifnum#1=1 \include{#3} 
	\else \ifincludealllabs \include{#3} \fi 
	\fi
	}
  {%There is a directory {#1} {#2} {#3} \par
	\ifnum#1=1 \subincludefrom*{#2} {#3} 
	\else \ifincludealllabs \subincludefrom*{#2} {#3} \fi 
	\fi
	}
}


%---------------------------------------------------------------------------------------------------
%Packages and new commands providing general utilities

\usepackage{graphicx}
\usepackage{wrapfig}
\usepackage{comment}
\usepackage{makeidx}
\usepackage{esvect}
\usepackage{amsmath} %uncommented by MT 5/2015, used in "E near charged rod"
\usepackage{amssymb}
\usepackage{mathtools} %added by MT 6/2015, for access to dcases environment in finding_v_from_e

\usepackage{tabularx} %added by MT 6/2015, for fixed width columns, used in rc_circuits
%For fixed width columns:
\newcolumntype{L}[1]{>{\raggedright\arraybackslash}p{#1}}
\newcolumntype{C}[1]{>{\centering\arraybackslash}p{#1}}
\newcolumntype{R}[1]{>{\raggedleft\arraybackslash}p{#1}}
% define a strut for extra vertical space in tables.
\newcommand{\hi}{\rule[-2mm]{0mm}{6mm}}

%I make use of some latex features to manage the section numbers. To use those you have to insert the following lines into the latex preamble (before the %"\begin{document}" command). -gpg
% two new commands to do labelling. - gpg 12/4/13
\newcommand{\customlabel}[2]{%
\protected@write \@auxout {}{\string \newlabel {#1}{{#2}{}}}}

\newcommand{\actlabel}[1]{%
\protected@write \@auxout {}{\string \newlabel {#1}{{\arabic{activity}}{}}}}

\newcounter{activity}

\usepackage{enumitem} %added 3/2016 by MT.  Tested 100% with other labs, looks like it does no harm.
%enumitem package allows enumerate[resume], which allows enumerate to play nicely with wrapfig (see induction1).
%enumitem package also provides very useful [wide] and [nosep] presets. (See, eg. resonance_tubes).
\SetEnumitemKey{labparts}{wide, label=(\emph{\alph*})} %defines labparts to number parts as (a), (b), etc.


\usepackage{chngcntr} %allows control over counters
\counterwithin*{footnote}{section} %resets the footnote counter to 1 at every new section (each lab)
\counterwithin*{equation}{section} %resets the equation counter to 1 at every new section (each lab)
\counterwithin*{figure}{section} %resets the figure counter to 1 at every new section (each lab)

\newcommand\specialcaret {\hspace{0.0ex}\raisebox{0.6ex}{\scalebox{0.7}[0.7]{$\boldsymbol{\wedge}$}}\hspace{0.1ex}}

\newcommand{\makelabheader}
%{Name: \rule{2.0in}{0.1pt}\hfill{}Section: \rule{1.0in}{0.1pt}\hfill{}Date: \rule{1.0in}{0.1pt}}
{Name: \rule{2.0in}{0.1pt}\hfill{}Lab Partner(s): \rule{3.0in}{0.1pt}}


%A new command for putting a box around a paragraph:
\newenvironment{newboxed} %maybe there's a better way to do this.  I just cribbed from the web. --MT
    {\begin{center}
    \begin{tabular}{|p{0.9\textwidth}|}
    \hline\\
    }
    { 
    \\\\\hline
    \end{tabular} 
    \end{center}
    }


%  The following command, \answerspace, should be used to replace \vspace.
%  \vspace{} is not ideal for an answer space for students, for two reasons:
%  1. It can be ignored if it comes at the end of a page, and
%  2. The spacing is exact, and Latex will not stretch or compress it at all to make things fit on a page, which means
%  that other things WILL get stretched or compressed to make things fit, which means those other things will 
%  end up looking bad, and leading to a lot of underfull \vbox warnings.
%  \answerspace fixes both of those problems, specifically allowing the space to grow to up to 1.5 times the stated size.
\newlength\answerlength
\newcommand{\answerspace}[1]{
	\setlength\answerlength{#1}
	\vspace*{1.0\answerlength plus 0.5\answerlength}}

%The following code can be used to make each lab start on odd numbered page (right hand side).
\let\NoExtraPageSection\section
\let\NoExtraPagePart\part
\newcommand{\ForceSectionOddPage}{
	\renewcommand{\section}{\cleardoublepage\NoExtraPageSection}
	\renewcommand{\part}{\cleardoublepage\NoExtraPagePart}
	}
\newcommand{\NoForceSectionOddPage}{
	\renewcommand{\section}{\NoExtraPageSection}
	\renewcommand{\part}{\NoExtraPagePart}
	}


\newif\ifForOneTwentyFive %is set to true in master.tex for PHYS 125.  Allows alternate versions of the labs.  Use with care!

\newcommand{\filename}[1]{{\sffamily\fontshape{sl}\selectfont #1}}
\newcommand{\button}[1]{{\fontshape{sl}\selectfont #1}}
 %all general latex packages, commands, and definitions now here.
\newcommand{\coursefolder}{Phys131} %This defines the place students will look for various files
\ForceSectionOddPage %This option makes each lab start on odd numbered page (right hand side).
\externaldocument{master131}

%syntax: \includeonly{lab1,lab2,lab3} with no spaces after the commas.
%\includeonly{biot_savart_law/biot_savart_law, charge_density/charge_density,eoverm/eoverm }
%DANGER: The includeonly statement will make a document that does NOT have sequential page numbers.

\newcommand{\supplementmark}{RELATIVITY}

\titleformat{\section}{\normalfont\Large\bfseries}{\supplementmark \thesection}{1em}{}
\fancyhead[LO,RE]{\slshape \rightmark} 
\fancyhead[LE,RO]{\slshape \supplementmark \leftmark} % Reversed LE, RO to  LO,RE to make headers come out correctly on even/odd 

\begin{document}

\setcounter{page}{1}  %Set this to desired first page
\setcounter{section}{0} %set this to desired first section number MINUS ONE

%--------------------------------------------
%Put include or includelab statements for labs below here.
\includelab{1}{galilean_relativity/galilean_relativity}
\includelab{1}{galilean_exercises/galilean_exercises}
\includelab{1}{time_dilation_length_contraction/time_dilation_length_contraction} 
\includelab{1}{time_dilation_length_contraction/muon_lifetimes}
\includelab{1}{lorentz_transformations/lorentz_transformations}
\includelab{1}{pole_and_barn/pole_and_barn} 
\includelab{1}{twins_paradox/twins_paradox}

%--------------------------------------------
\appendix
%\setcounter{section}{4} %set this counter to number MINUS ONE corresponding to desired appendix letter. (4 for `E', etc.)
%Put include statements for supplementary appendices below here.
%\includelab{1}{appendices/nuke_safety/nuke_safety}

% The following command prints the "Instructor Notes" section at the end of the manual.
% Comment it out for the regular student edition.
\startinstructornotes
\end{document}
