\section{The Period of a Pendulum}

\instructornote{%
By Matt Trawick, 2019.  Time: 60-70 minutes.

A major slowdown on this lab was Activity 2(e), where students had to use LINEST and also calculate an uncertainty in $g$.  Telling student to save that part for homework would shave off significant class time, if you choose to do so.  The obvious way to calculate $\delta g$ is to use the uncertainty generated by LINEST.  This will generally underestimate the error, which is okay, but could presumably prompt a conversation to remind the class that LINEST only estimates random errors, not systematic errors.

The introduction of the lab gives the period for the simple pendulum, but does not explain why or derive it.  Presumably that needs to be done ahead of time.  (Note by Matt: I started out by just deriving the equation of motion, from $\tau=I\alpha$:
$$
(mg)(L)\sin \theta = (mL^2)\frac{d^2 \theta}{dt^2},
$$
then I asked them if they thought the period would increase or decrease with mass, \textit{before} I actually canceled out the $m$'s on the whiteboard.  Then I let them do activity 1, and then I interrupted them with the rest of a derivation of $T$ for small angles.  Your mileage may vary.)

Equipment notes:

Start by adjusting the string height so that there is enough clearance for the largest (500 g) mass.

The guide hole for the string should actually be turned about 45 degrees, not straight vertical.
}

\makelabheader %(Space for student name, etc., defined in master.tex or labmanual_formatting_commands.tex)

\bigskip
\textbf{Objectives:}

To study the influences on the motion of a simple pendulum, and to calculate the gravitational acceleration $g$ from measurements of the period and length of a simple pendulum.

\bigskip
\textbf{Apparatus:}
\begin{itemize}[nosep]
\item Collection of masses 
\item String, attached to lab stand scaffolding
\item Protractor, also attached to lab stand scaffolding
\item Photogate
\item \textit{Capstone} software (\filename{pendulum.cap} experiment file)
\item Meter stick
\end{itemize}


\bigskip
\textbf{Introduction:}

The period of a simple pendulum is related to its length $L$ and the acceleration due to gravity $g$ according to the relationship:
\begin{equation}
T=2\pi \sqrt{\frac{L}{g}}.
\end{equation}

This assumes the oscillations are small. Let's check this prediction experimentally. 

\bigskip
\textbf{Activity 1: The Effect of Mass on the Period} 

\begin{enumerate}[labparts]

\item Make a prediction: if you increase the mass of a pendulum, will the pendulum's period \textit{increase}, \textit{decrease}, or \textit{stay the same}?  Why do you think so?
\answerspace{0.8in}

To test your prediction, you will use a photogate to measure the period of your pendulum.  
Open the file \filename{pendulum.cap} in the \filename{\coursefolder} folder.  
Adjust the string guide on your lab stand to its maximum height, so that the effective length of the pendulum is about 1~meter. 

\item For each mass, measure the periods for several oscillations, and record a typical value in the table below.  Use a constant amplitude of not greater than 10 degrees.  Also, because the weights are slightly different sizes, you will need to readjust the string guide up or down slightly for each weight, so that the length $L$ from the pivot point to the weight's approximate center of mass remains constant.  

\begin{center}
{\renewcommand{\arraystretch}{1.8}
\begin{tabular}{|c|c|c|c|c|} \hline 
Trial No. & Mass (kg) & Length $L$ (m) & Amplitude $\theta_{\rm max}$ ($^\circ$) & Period $T$ (s) \\ 
\hhline{|=|=|=|=|=|}
1 & 0.050 & & & \\ \hline 
2 & 0.200 & & & \\ \hline 
3 & 0.500 & & & \\ \hline 
\end{tabular} }
\end{center}

\item Was your prediction correct?  Does the period of the pendulum appear to depend on mass?
\answerspace{0.8in}

\end{enumerate}

\textbf{Activity 2: The Effect of Length on the Period} 

\begin{enumerate}[labparts]

\item Make a prediction: if you increase the length $L$ of a pendulum, will the pendulum's period \textit{increase}, \textit{decrease}, or \textit{stay the same}?  Why do you think so?
\answerspace{0.8in}

\item To test your prediction, use a 100~g or 200~g mass, and adjust the height of the string guide on your apparatus to change the effective length $L$ of your pendulum to several different values.  Remember to record the length $L$ from the pivot point to the center of mass of the hanging weight.  


\begin{center}
{\renewcommand{\arraystretch}{1.8}
\begin{tabular}{|c|c|c|c|c|c|c|} \hline
Trial No. & Mass (kg) & Length $L$ (m) & Amplitude $\theta_{\rm max}$ ($^\circ$) & Period $T$ (s) \\ 
\hhline{|=|=|=|=|=|}
1 & & & & \\ \hline 
2 & & & & \\ \hline 
3 & & & & \\ \hline 
4 & & & & \\ \hline 
5 & & & & \\ \hline
\end{tabular} }
\end{center}

\item Plot $T$ as a function of $L$ from the table above on a log-log graph, and fit the data to a power law.  Print the graph and include it with this unit.  Is the power law you found consistent with the theoretical prediction of Equation (1)?
\answerspace{0.6in}

\item You should have found that the period $T$ is proportional to $L^{1/2}$.  From Equation 1, what is that constant of proportionality?
\answerspace{0.4in}

\pagebreak[2]
\item Use your data to estimate the value of gravitational acceleration $g$, along with an uncertainty of that estimate.  To obtain a value for the uncertainty in your proportionality constant, you will use the LINEST function of Excel, as described in Appendix \ref{excel}.
But to make LINEST fit your $T$ values to $L^{1/2}$ instead of $L$, you will enter ``\specialcaret \verb!0.5!'' after selecting your $L$ data (what Excel calls the ``known $x$ values'') so that your entire line will look something like ``\verb!=linest(B1:B5, A1:A5! \specialcaret \verb!0.5, 0, 1)!''.  Write your value of $g$, with uncertainty, as $g \pm \delta g$ below.
\answerspace{1.4in}
\end{enumerate}

\textbf{Activity 3: The Effect of Amplitude on the Period} 

Throughout this lab, we have noted that Equation (1) is only valid for ``small'' oscillations.  In this activity, we will see what effect amplitude really has on the period. 

\begin{enumerate}[labparts]
\item Record the period $T$ for several different amplitudes $\theta_{\rm max}$.  Use a mass of 100~g or 200~g, and maintain a constant value of $L \approx 1$~m. Any differences you see will be small, so you will need to take careful measurements.  Use a mass with a hook that is not crooked, so that the mass hangs straight and the timing won't be affected if the mass spins.  Also, record several periods, and calculate the average and standard deviation of the period for each amplitude using the statistics function in Capstone. (See Appendix \ref{capstone}.  Also, if you have a few obviously bogus data points at the beginning or end, you may need to select only the points you want using the yellow highlighting tool.)  Try to take data for amplitudes of $1^\circ, 2^\circ , 4^\circ , 8^\circ , 16^\circ , 24^\circ ,$ and $32^\circ$.

\begin{center}
{\renewcommand{\arraystretch}{1.8}
\begin{tabular}{|c|c|c|c|C{0.8in}|C{0.8in}|} \hline
Trial No. & Mass (kg) & Length $L$ (m) & Amplitude $\theta_{\rm max}$ ($^\circ$) & $\langle T \rangle$ (s) & $\sigma T$ (s) \\ 
\hhline{|=|=|=|=|=|=|}
1 & & & & & \\ \hline 
2 & & & & & \\ \hline 
3 & & & & & \\ \hline 
4 & & & & & \\ \hline 
5 & & & & & \\ \hline 
6 & & & & & \\ \hline 
7 & & & & & \\ \hline 
\end{tabular} }
\end{center}

\item Does your data suggest that the period is independent of amplitude $\theta_{\rm max}$ for ``small'' amplitudes?
\answerspace{0.5in}

\item About how big does the amplitude $\theta_{\rm max}$ need to be for the period to change by 1\%?
\answerspace{0.6in}

\end{enumerate}
