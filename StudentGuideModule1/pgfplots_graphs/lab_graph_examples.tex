\documentclass{article}
\usepackage{pgfplots}
\usepackage{sansmath}
\usepgfplotslibrary{groupplots}
\usetikzlibrary{pgfplots.groupplots}
\usetikzlibrary{math}
\usepackage{xparse} %used for \NewDocumentCommand
\listfiles

\pgfplotsset{compat=1.16}
\tikzstyle{every picture}+=[font=\sffamily]

\pgfplotscreateplotcyclelist{lab_solid}{
	{black}
}

\pgfplotsset{
  tick label style = {font=\sansmath\sffamily},
%  every axis label = {font=\sansmath\sffamily},
%  legend style = {font=\sffamily},
%  label style = {font=\sffamily}
	samples = 200,
}


\pgfplotsset{lab_noticks_1quad/.style={
	cycle list name = lab_solid,
	every axis plot/.append style={very thick},
	axis x line={left, x axis line style={-}}, 
	axis y line= {left,  y axis line style={-}},
	xmin=0, xmax=1,
	ymin=0, ymax=1, %Although there are no ticks, this seems to be needed to make the ticks disappear.
	tick style={major tick length=0pt},
	tick style={xtick=\empty,ytick=\empty}
}}

\pgfplotsset{lab_noticks_2quads/.style={
	cycle list name = lab_solid,
	every axis plot/.append style={very thick},
	axis x line= {middle, x axis line style={-}}, 
	axis y line= {left,  y axis line style={-}},
      xlabel style={at=(current axis.right of origin), anchor=west},
	xmin=0, xmax=1,
	ymin=-1, ymax=1, 
	tick style={major tick length=0pt},
	tick style={xtick=\empty,ytick=\empty},
}}

\pgfplotsset{lab_noticks_4quads/.style={
	cycle list name = lab_solid,
	every axis plot/.append style={very thick},
	axis x line= {middle, x axis line style={-}}, 
	axis y line= {middle,  y axis line style={-}},
      xlabel style={at=(current axis.right of origin), anchor=north east},
      ylabel style={at=(ticklabel* cs:1.0), anchor=north east},
	xmin=-1, xmax=1,
	ymin=-1, ymax=1, 
	tick style={major tick length=0pt},
	tick style={xtick=\empty,ytick=\empty},
}}

\pgfplotsset{plus_minus_zero_labels/.style={
	extra y ticks = {-0.8, 0, 0.8},
	extra y tick labels = {\scriptsize{$-$},0,\scriptsize{$+$}},
}}

\pgfplotsset{lab_grid/.style={
	cycle list name = lab_solid,
	every axis plot/.append style={very thick},
	tick style={grid=major,major tick length=0pt}
}}

\newcommand{\oldmakegroupverticals}[2]{
	\coordinate (toppoint) at  (group c1r1.north);
	\coordinate (botpoint) at  (group c1r#1.south);
	\foreach \x in {#2} 
		\draw [black, very thick, dashed] (\x, |- toppoint) -- (\x, |- botpoint);
}

\NewDocumentCommand \makegroupverticals {o m} {
	\coordinate (toppoint) at  (group c1r1.north);
      \IfNoValueTF  {#2} {%If no lastrow given, assume two rows
		\coordinate (botpoint) at  (group c1r2.south);
		}
	{ % else use the value for the number of the last row
		\coordinate (botpoint) at  (group c1r#1.south);
	}
	\foreach \x in {#2} 
		\draw [black, very thick, dashed] (\x, |- toppoint) -- (\x, |- botpoint);
}



\usepackage[parfill]{parskip}

\begin{document}

A blank set of axes, in one quadrant, with labels:

\begin{lab_axis}[lab_noticks_1quad,
	height = {1.7in}, width = {2in},
	xlabel={time (s)},
	ylabel={position (m)},
	]
\end{lab_axis}

Two quadrants, with a graph, centered:

\begin{lab_axis*}[lab_noticks_2quads,
	xlabel={time (s)},
	ylabel={velocity (m/s)},
	]
\addplot coordinates {(0, 0.6) (0.9, 0.6)};
\end{lab_axis*}

With some extra tick marks:

\begin{lab_axis}[lab_noticks_2quads,
	xlabel={time (s)},
	ylabel={velocity},
	xtick={0.2,0.5,0.9},
	xticklabels={A,B,C},
	ytick={0.2,0.7},
	yticklabels={min,max},
	]
\end{lab_axis}

Two quadrants with sign labels, and several labeled curves:

\begin{lab_axis}[lab_noticks_2quads,
	xlabel={time (s)},
	ylabel={position (m)},
	plus_minus_zero_labels,
	]
\addplot coordinates {(0,0.1) (0.9,0.8)} node[right] {A};
\addplot coordinates {(0,0.9) (0.8,0.1)} node[right] {C};
\addplot[domain=0:0.9] {0.6*x^2 + 0.1} node[right] {B};
\end{lab_axis}

\newpage

And, finally, four quadrants:

\begin{lab_axis}[lab_noticks_4quads,
	height = {2in}, width = {2.9in},
	xlabel={$x$},
	ylabel={$F(x)$},
	]
\addplot coordinates {(-0.9, 0.8) (0.9, -0.8)};
\node[anchor=west,text width = 1.5in] at (axis cs: 0.1,0.5)  {Linear spring: \\ $F(x) = -k(x)$};
\end{lab_axis}
\begin{lab_axis}[lab_noticks_4quads,
	width = {2.5in},
	xlabel={$x$},
	ylabel={$U(x)$},
	ymin=-0.2,
	]
\addplot [domain=-0.9:0.9] {x^2};
\node[anchor=west] at (axis cs: 0.0,0.5)  {$U = \frac{1}{2}kx^2$};
\end{lab_axis}


\bigskip
I also defined a style that includes a grid: 

\begin{lab_axis}*[lab_grid,
	height = {2in},
	width = {4in},
	xlabel={time (s)},
	ylabel={position (m)},
	xmin=0, xmax=30,
	ymin=-2, ymax=2,
	xtick distance = 5,
	ytick distance = 1,
	y0_line,
	]
\addplot coordinates {(0,0) (5,0) (5,1) (10,1) (10,0) (15,0) (15,-1) (20,-1) (20,0) (25,0)};
\end{lab_axis}

You can also add minor tick marks to it: 

\begin{lab_axis}*[lab_grid,
	height = {2in},
	width = {4in},
	xlabel={time (s)},
	ylabel={position (m)},
	xmin=0, xmax=30,
	ymin=-2, ymax=2,
	xtick distance = 5,
	ytick distance = 1,
	minor tick num=1,
	minor x tick num=4,
%	minor tick num=1, %use this for both x and y
	]
\end{lab_axis}

\newpage

And you can make various labels disappear: 

\begin{lab_axis}[lab_grid,
	xlabel={time (s)},
	ylabel={position (m)},
	xmin=0, xmax=20,
	ymin=-2, ymax=2,
	extra y ticks={0},  %use this, or my definition "y0_line" for black line at y=0
	]
\end{lab_axis}
\begin{lab_axis}[lab_grid,
	xlabel={time (s)},
	ylabel={position (m)},
	xmin=0, xmax=20,
	ymin=-2, ymax=2,
	yticklabels = { , , },
	y0_line,
	]
\end{lab_axis}

\begin{lab_axis}[lab_grid,
	xlabel={time (s)},
	ylabel={position (m)},
	xmin=0, xmax=20,
	ymin=-2, ymax=2,
	yticklabels = { , , },
	extra tick style={yticklabels= { , , }},
	]
\end{lab_axis}
\begin{lab_axis}[lab_grid,
	xlabel={time (s)},
	ylabel={position (m)},
	xmin=0, xmax=20,
	ymin=-2, ymax=2,
	yticklabels = { , , },
	extra tick style={yticklabels= { , , }},
	xticklabels = { , , },
	]
\end{lab_axis}

Note that the "groupplot" feature can be used to stack graphs:

\begin{lab_groupplot}*{}[
 	group style={
      		group size=1 by 2,
		xlabels at=edge bottom,
		xticklabels at=edge bottom,
		vertical sep=0.2in
		},
	height=2in, width=2.5in,
	]
\nextgroupplot[
	lab_noticks_1quad,
	xlabel={time (s)},
      xlabel style={at=(current axis.right of origin), anchor=west},
	ylabel={position (m)},
	]
\nextgroupplot[
	lab_noticks_2quads,
	plus_minus_zero_labels,
	xlabel={time (s)},
	ylabel={velocity (m/s)},
	]
\end{lab_groupplot}

Finally, you can drop alignment marks between graphs, as the next two examples show:

\begin{lab_groupplot}*{\makegroupverticals[2]{1, 3, 5, 7}{0}{9}}
					[lab_grid,
	group style={
		group size=1 by 2,
		xlabels at=edge bottom,
		xticklabels at=edge bottom,
		vertical sep=0.2in
		},
	width=4.5in,
	height=2in,
	xlabel=time (s),
	xmin=0, xmax=9,
	]
\nextgroupplot[ymax=15, ylabel={position (m)}]
%\addplot coordinates{(0,1) (1,1)};
\addplot coordinates{(0,1) (1,1) (3,5) (5,5) (7,13) (8,13)};

\nextgroupplot[ymin=0,ymax=5, minor y tick num=1, ylabel={velocity (m/s)}]
\addplot coordinates{(0.5,0.5)};
%Weird!!! Note that the nonfunctional addplot line just above is needed, even though it makes no actual mark.
%Without the addplot, with at least ONE point somewhere in the visible area, the scale of the x axis is totally wrong.

%\addplot coordinates{(0,0) (1,0) (1,2) (3,2) (3,0) (5,0) (5,4) (7,4) (7,0) (8,0)};
\end{lab_groupplot}

\begin{lab_groupplot}{\makegroupverticals[3]{90,180,270,360,450,540,630,720}{0}{770}}
					[lab_noticks_2quads,
	xmin=0,xmax = 770,
 	group style={
		group size=1 by 3,
		vertical sep=0.2in
		},
	width=4in,
	height=2in,
	xlabel=time $t$,
	]
\nextgroupplot[ylabel={current $I(t)$}]
\addplot [domain=0:720] {0.90*sin(x) }; %Why is it calculating sine in degrees?

\nextgroupplot[ylabel={flux $\Phi(t)$}]

\nextgroupplot[ylabel={emf $\varepsilon(t)$}]

\end{lab_groupplot}

\end{document}
