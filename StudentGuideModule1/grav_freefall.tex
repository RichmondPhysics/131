
\section{Gravity and Free Fall\footnote{
1990-93 Dept. of Physics and Astronomy, Dickinson College. Supported by FIPSE
(U.S. Dept. of Ed.) and NSF. Portions of this material have been modified locally
and may not have been classroom tested at Dickinson College.
}}

\makelabheader %(Space for student name, etc., defined in master.tex or labmanual_formatting_commands.tex)

\textbf{Objectives }

To explore the phenomenon of gravity and study the nature of motion along a
vertical line near the earth's surface.

\textbf{Overview }

When an object falls close to the surface of the earth, there is no obvious
force being applied to it. Whatever is causing it to move is invisible. Most
people casually refer to the cause of falling motions as the action of 
``gravity.''
What is gravity? Can we describe its effects mathematically? Can Newton's laws
be interpreted in such a way that they can be used for the mathematical prediction
of motions that are influenced by gravity? In this investigation we will study
the phenomenon of gravity for vertical motion. You will need:

\textbf{Apparatus}

\begin{itemize}
\item A tennis ball. 
\item A movie scaling ruler.
\item A video analysis system (\textit{VideoPoint} or \textit{Tracker}). 
\item Graphing software (\textit{Excel}).
\end{itemize}
\textbf{Vertical Motion: Describing How Objects Rise and Fall }

Let's begin the study of the phenomenon of gravity by predicting the nature
of the motion of an object, such as a tennis ball, when it is tossed up and
then allowed to fall vertically near the surface of the earth. This is not easy
since motion happens pretty fast! To help you with this prediction you should
toss a ball in the laboratory several times and see what you think is going
on.

\textbf{Activity 1: Predicting the Motion of a Tossed Ball }

(a) Toss a ball straight up a couple of times and then describe how you think
it might be moving when it is moving upward. Some possibilities include: (1)
rising at a constant velocity; (2) rising with an increasing acceleration; (3)
rising with a decreasing acceleration; or (4) rising at a constant acceleration. What do you think? (You may want to review Activity 7 of Experiment 7.)
\vspace{20mm}

(b) Explain the basis for your prediction.
\vspace{20mm}

(c) Now describe how you think the ball might be moving when it is moving downward. Some possibilities include: (1) falling at a constant velocity; (2) falling
with an increasing acceleration; (3) falling with a decreasing acceleration;
or (4) falling at a constant acceleration. What do you think?
\vspace{20mm}

(d) Each group in the laboratory will measure the acceleration of a ball. Should these different measurements be the same? Why or why not?
\vspace{20mm}

(d) Explain the basis for your prediction.
\vspace{20mm}

(e) Do you expect the acceleration when the ball is rising to be different in
some way from the acceleration when the ball is falling? Why or why not?
\vspace{20mm}

(f) What do you think that the acceleration will be at the moment when the ball
is at its highest point? Why?
\vspace{20mm}

The motion of a tossed ball is too fast to observe carefully by eye without
the aid of special instruments. In the next activity we will use a video analysis system to study the motion of a freely falling object. You will use a sequence of these video frames and mathematical modeling techniques to find an equation that describes the fall. 

\textbf{Activity 2: Analyzing the Motion of a Tennis Ball} 

(a) Make a movie of a tennis ball in flight by following these steps.

\begin{enumerate}
\item Turn on the video camera and center the field of view on the region where you will toss the ball. This region should be about 2 meters from the camera to
get a large enough area for the flight of the ball. Place a ruler or meter stick somewhere in the field of view where it won't interfere with the motion. This
ruler will be used later to determine the scale. 
\item Make a movie of the tennis ball being dropped from rest. Make sure most of the trajectory is visible to the camera. See \textbf{Appendix \ref{videopoint}: Video Analysis Using VideoPoint} or \textbf{Appendix \ref{tracker}: Video Analysis Using Tracker} for details on making the movie.
%When you save the movie file give it the name
%Ball.
\end{enumerate}
\textbf{IMPORTANT:} Do NOT save the movie to your netfiles space.  Save it to the DESKTOP.  (Before logging out later you can save the movie file to your own space on Saturn.)
\vspace{5mm}

(b) Determine the vertical position, $y$, of the tennis ball at different times
during the motion. If using \textbf{VideoPoint}, follow the instructions in the second section of \textbf{Appendix \ref{videopoint}: Video Analysis - Analyzing the Movie}. 

(c) Use \textit{Excel} to plot a graph of $y$ vs. $t$. See \textbf{Appendix
\ref{excel}: Introduction to Excel} for details.

(d) What is the initial value of $y$ (usually denoted \( y_{0} \))(from your data table)? 
\vspace{10mm}

(e) By examining your data table, calculate the approximate value of the initial velocity of the ball in the $y$-direction (from the first two lines of data). Include the sign of the velocity and its units. (Use the convention that on the $y$-axis up is positive and down is negative.)
\vspace{20mm}

(f) Examine the graph of your data. What does the nature of this motion look
like? Constant velocity, constant acceleration, an increasing or decreasing
acceleration? How does your observation compare with the prediction you made
earlier in this unit for the ball on its way down?
\vspace{20mm}

(g) Using the convention that on the $y$-axis up is positive and down is negative, is the acceleration positive or negative (i.e., in what direction is the magnitude of the velocity increasing)?
\vspace{20mm}

(h) If you think the object is undergoing a constant acceleration, use the fitting capability of \textit{Excel} (see \textbf{Appendix \ref{excel}: Introduction to
Excel} for details) to find an equation that describes $y$ as a function
of $t$ as the ball drops. Hints: (1) You might try to model the system with a
second order equation like the kinematic equation for uniformly accelerated
motion. (2) Write the equation of motion in the space below. Then use coefficients of the best-fit equation to find the values of $a$, \( v_{0} \) and \( y_{0} \) with the appropriate units. \textbf{Note: The numbers from your graph should be rounded off to no more than 3 significant figures}. When you have found a good fit to the data, print it and attach a copy to this unit. 

\begin{enumerate}
\item The equation of motion with proper units is: $y =$\vspace{5mm}

\item The acceleration with proper sign and units is: $a =$ \vspace{5mm}

\item The initial velocity with proper sign and units is: \(v_{0} \) = \vspace{5mm}

\item The initial position with proper sign and units is: \(y_{0} \) = \vspace{5mm}

\item The initial position and velocity of the ball depend on the details of your throwing motion. 
The acceleration does not.
It depends only on the motion of ball after you have released it.
Go around to the other groups in the lab and ask them for the value of the acceleration they obtained.
Make a histogram of your results and calculate the average and standard deviation of the acceleration for the whole class.
For information on making histograms, see \textbf{Appendix \ref{excel}}. For information on calculating the average and
standard deviation, see \textbf{Appendix \ref{treatment}}. Record the average and standard deviation here.
Attach the histogram to this unit.
\vspace{20mm}

\item Is the acceleration of the ball the same for the entire class? Use the average and standard deviation for the class to quantitatively answer this question.
\vspace{20mm}

\item What does the histogram of the class data tell you? Be quantitative in your answer.
\vspace{20mm}

\end{enumerate}
