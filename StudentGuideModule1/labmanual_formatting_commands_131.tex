\begin{comment}
Note by Matt Trawick, 10/29/2015:
The contents of this file were removed from inside the 131 master.tex file.  The 131 master.tex file now reads the file labmanual_formatting_commands.tex, where all of the packages and latex commands are defined.  That file, labmanual_formatting_commands, was copied from the 132 folder.  The 132 version included all of the stuff here, plus some other stuff that won't hurt.  This file is probably useless, but is being retained here just in case, for now.
\end{comment}

%% This LaTeX-file was created by <vineyard> Fri Jul 28 09:26:18 2000
%% LyX 1.0 (C) 1995-1999 by Matthias Ettrich and the LyX Team

%% Do not edit this file unless you know what you are doing.
\documentclass[twoside]{article}
\usepackage[T1]{fontenc}
\usepackage{geometry}
\geometry{verbose}
\usepackage{fancyhdr}
\pagestyle{fancy}
\setlength\parskip{\medskipamount}
\setlength\parindent{0pt}
\usepackage{graphicx}
\usepackage{comment}

% original values
% \setlength\topmargin{0.25in}
% \addtolength{\hoffset}{-1.0cm}
% \addtolength{\textwidth}{2.0cm}
% \addtolength{\voffset}{-1.5cm}
% \addtolength{\textheight}{3.5cm}
%
% new ones.
\setlength\topmargin{0.2in}
\addtolength{\hoffset}{-1.0cm}
\addtolength{\textwidth}{2.0cm}
\addtolength{\voffset}{-1.5cm}
\addtolength{\textheight}{3.5cm}

%The \includonly line below is a great way to save time so you don't always have to compile the WHOLE latex document, if for instance you've only made changes to a single lab.  If you want to compile more than two labs, the syntax is \includeonly{lab1,lab2,lab3} with no spaces after the commas.
%The master.pdf produced will have only the title page, TOC, and that single lab, though the other lab names will appear in the TOC.
%\includeonly{newtons_laws/newtons_laws}

\makeatletter


%%%%%%%%%%%%%%%%%%%%%%%%%%%%%% LyX specific LaTeX commands.
\providecommand{\LyX}{L\kern-.1667em\lower.25em\hbox{Y}\kern-.125emX\@}
%% Special footnote code from the package 'stblftnt.sty'
%% Author: Robin Fairbairns -- Last revised Dec 13 1996
\let\SF@@footnote\footnote
\def\footnote{\ifx\protect\@typeset@protect
    \expandafter\SF@@footnote
  \else
    \expandafter\SF@gobble@opt
  \fi
}
\expandafter\def\csname SF@gobble@opt \endcsname{\@ifnextchar[%]
  \SF@gobble@twobracket
  \@gobble
}
\edef\SF@gobble@opt{\noexpand\protect
  \expandafter\noexpand\csname SF@gobble@opt \endcsname}
\def\SF@gobble@twobracket[#1]#2{}

\makeatother
