
\section{Converting Units}

Name \rule{2.0in}{0.1pt}\hfill{}Section \rule{1.0in}{0.1pt}\hfill{}Date \rule{1.0in}{0.1pt}

{\noindent \bf Objectives:} \begin{list}{$\bullet$}{\itemsep0pt \parsep0pt}

\item Introduction to units \item Exploration of relations between different systems of measurement

\end{list}

{\noindent \bf Apparatus:} \begin{list}{$\bullet$}{\itemsep0pt \parsep0pt}

\item string \item scissors \item meter stick

\end{list}

{\noindent \bf Activity:} \begin{enumerate}

\item Cut the string into three pieces, 1-meter, 1-foot, and 30-cm, respectively.

\item Compare them. Enter into the table below how many lengths of one piece it requires to make each of the others.

\item Cut a length of string equivalent to your anatomical foot; compare this piece to the others.

\begin{center} \begin{tabular}{|l|c|c|c|c|c|} \hline \multicolumn{2}{|c||}{} & \multicolumn{4}{c|}{\bf equal one of these?} \\ \hline \multicolumn{1}{|c}{}& \multicolumn{1}{c||}{pieces}& 1-meter &1-foot & 30-cm & anatomical foot \\ \hline \hline {\bf How} & \multicolumn{1}{c||}{1-meter} &&&& \\ \cline{2-6} {\bf many} & \multicolumn{1}{c||}{1-foot} &&&& \\ \cline{2-6} {\bf of} & \multicolumn{1}{c||}{30-cm} &&&& \\ \cline{2-6} {\bf these} & \multicolumn{1}{c||}{anatomical foot} &&&& \\ \hline \end{tabular} \end{center}

\item Devise your own measurement system; name it; select length, mass, and time standards and choose names and abbreviations for them; and define these.

\begin{center} system name \rule{3in}{0.2pt} \end{center} \begin{tabbing} length: \= name \rule{1in}{0.2pt} \= abbreviation \rule{0.5in}{0.2pt} \= definition \rule{1.5in}{0.2pt} \\ mass: \> name \rule{1in}{0.2pt} \> abbreviation \rule{0.5in}{0.2pt} \> definition \rule{1.5in}{0.2pt} \\ time: \> name \rule{1in}{0.2pt} \> abbreviation \rule{0.5in}{0.2pt} \> definition \rule{1.5in}{0.2pt} \end{tabbing}

\item Compare your measurement system to the Standard International (MKSA) system.

\begin{tabbing} 1 \ \rule{1in}{0.2pt} \= = \= \rule{0.25in}{0.2pt} \ \= meters \kill 1 meter \> = \> \rule{0.25in} {0.2pt} \ \> \rule{1in}{0.2pt}s \\ 1 kilogram \> = \> \rule{0.25in} {0.2pt} \ \> \rule{1in}{0.2pt}s \\ 1 second \> = \> \rule{0.25in} {0.2pt} \ \> \rule{1in}{0.2pt}s \\ 1 \ \rule{1in}{0.2pt} = \rule{0.25in}{0.2pt} \ meters \\ 1 \ \rule{1in}{0.2pt} = \rule{0.25in}{0.2pt} \ kilograms \\ 1 \ \rule{1in}{0.2pt} = \rule{0.25in}{0.2pt} \ seconds \end{tabbing}

\end{enumerate}

\pagebreak

\textbf{Questions: }

1. Are anatomical units practical for measurement? Explain. 
\vspace{20mm}

2. What criteria should be satisfied by a good clock? 
\vspace{20mm}

3. Can you explain the nearly universal use of the metric system in scientific
work?

