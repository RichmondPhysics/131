
\section{Conservation of Angular Momentum}

\makelabheader %(Space for student name, etc., defined in master.tex or labmanual_formatting_commands.tex)

\textbf{Objectives} 

To test the Law of Conservation of Angular Momentum and to explore the applicability
of angular momentum conservation among objects that experience no external torques. 

\textbf{Apparatus}

\begin{itemize}
\item A Rotating Disk System 
\item Masses of 500 g, 1 kg 
\item A movie scaling ruler and a simple cm ruler 
\item A vernier caliper
\item A small water bubble level
\item A video analysis system (\textit{Tracker})
\end{itemize}
\textbf{Overview }

As a consequence of Newton's laws, angular momentum like linear momentum is
believed to be conserved in isolated systems. This means that, no matter how
many internal interactions occur, the total angular momentum of any system should remain constant if there are no external torques. When one of the objects gains some angular momentum another part of the system must lose the same amount. If angular momentum isn't conserved, then we believe that there is some outside torque acting on the system. By expanding the boundary of the system to include the source of that torque we can always preserve the Law of Angular Momentum Conservation. 

In this unit you will test the notion of the conservation of angular momentum.
As in the test of the conservation of linear momentum, we will investigate what
happens when two bodies undergo a ``rotational'' collision.
You will drop a large weight onto a rotating disk and determine the moment of
inertia, the angular speed, and finally, the angular momentum of the rotator-disk-weight
system before and after this perfectly inelastic collision.

\textbf{Activity 1: The Moment of Inertia Before and After the Collision}

(a) Calculate the theoretical value of the moment of inertia of the disk
using basic measurements of its radius and mass. Be sure to state units and
show the expression you used!
\vspace{5mm}

\( R_{d} =\)  \hfill{}\( M_{d}= \) \hfill{}
\vspace{5mm}

\( I_{d}= \)
\vspace{5mm}

(b) After dropping the weight on the rotating disk, the system will have a new
moment of inertia. Derive a formula for the moment of inertia of the weight (mass $m_w$) revolving about the origin at a distance $r_r$. You can treat the weight like a point particle.\footnote{Of course the weight isn't actually a point particle.  For a really precise calculation, you would also need to account for its shape and its rotation about its own center of mass, using the \textit{parallel axis theorem}.}  Also record the mass of the weight, $m_w$.
\vspace{5mm}

$m_w=$ \hfill{} $I_w =$ \hfill{}  
\vspace{10mm}

%(c) Record the mass of one of the weights and use a vernier caliper to measure its diameter. Obtain its radius from the diameter.
%\vspace{5mm}

%\( m_{w} =\)  \hfill{}\(r_{w} =\) \hfill{}
%\vspace{5mm}

\pagebreak[2]
(c) Come up with a formula for the moment of inertia $I$ of the whole system
before and after the collision and calculate the moment of inertia before the
collision only. (The moment of inertia after the collision will be determined AFTER you do the experiment, because you will not know \( r_{r} \) until after you do the experiment.) Don't forget to include the units in \( I_{before} \).
\vspace{5mm}

\( I_{\mbox{\small before}}= \) 
\vspace{5mm}

\( I_{\mbox{\small after}} =\)  
\vspace{5mm}

\textbf{Activity 2: Making a Movie of the Collision} 

(a) Place the video camera about 1 m above the rotator and align the camera with
the center of the rotator. Open \textbf{Movie Maker} and turn on the camera as explained 
in \textbf{Appendix \ref{tracker}: Video Analysis Using Tracker}.  Center the rotator in the
field of view of the camera. Place a ruler of known length in the field of view of the camera and parallel to one side of the frame. Check that the rotator is level with the small water-bubble level.

(b) Give the rotator a push and begin recording its motion with the video camera. See \textbf{Appendix \ref{tracker}: Video Analysis Using Tracker} for details on making the movie. 
While the rotator is moving hold the weight you recorded above near the rim of the disk and 
close to, but not quite touching, the surface of the moving disk. After at least one revolution
of the disk drop the weight onto the disk and record the motion of the disk for at least one 
revolution afterward. Do not rotate the disk too fast or the weight will not stay on it.

(c) \textit{Before removing the weight from the disk}, determine the distance of the center of the weight you dropped from the center of the rotator $r_w$. 
%To do this, measure the distance from the center of the rotator to the edge of the weight \( r_{edge} \) and use the result from Activity 1 part (c) for the radius of the weight \( r_{w} \). Calculate the distance from the origin to the center of the weight \( r_{r} \) (it is just the sum of \( r_{edge} \) and \( r_{w} \)). 
Use this result and your expression from Activity 1 part (c) to calculate the final moment of inertia.
\answerspace{5mm}

%\( r_{\mbox{\small edge}}  =\) \hfill{}\( r_{w} \) = \hfill{}\( r_{r} \) =\hfill{}
$r_w=$ \hfill{} $I_{after} =$ \hfill{}  
\answerspace{5mm}

%\( I_{\mbox{\small after}} =\)  
%\vspace{5mm}

\textbf{Activity 3: Measurement of Angular Velocity}

Determine the angular speed before and after the collision.

\begin{enumerate}
%\item Find the last frame before you dropped the weight on the rotator and click on
%the position of the white marker on the metal disk. Now go backward through the film until the rotator has gone through one full rotation. Estimate how many frames there are in one rotation. (Step through the frames and count.) This will be called \( N_{\rm before}\). You also need to know the time between frames \( \Delta  t_{\rm frame}\), which is just the reciprocal of the capture rate in frames per second.
%\[
%N_{\mbox{\small before}}=\qquad \qquad \qquad \Delta t_{\rm frame}=\]
\item Using \textit{Tracker}, track the outer edge of the white marker on the rotating disk 
for one full rotation before dropping the weight, and determine the period of rotation from the 
\textit{Tracker} graph. Record the result here and calculate the corresponding angular velocity 
in radians per second:
\[
T_{\mbox{\small before}}=\qquad \qquad \qquad \omega _{\mbox{\small before}}=\]
\vspace{5mm}
\item Follow a similar procedure to determine the angular speed after the collision. You will have 
to create another ``Point Mass'' in \textit{Tracker} in order to track the outer edge of the white 
marker after dropping the weight. Determine the period of rotation and the angular velocity as 
before and record the results here:
%Under the \textbf{Edit} Menu highlight \textbf{Clear All...} to get rid of your%
%previous results. 
%Find the first frame after you dropped the weight on the rotator
%and click on the position of the white marker. Now go forward through the film and estimate the number of frames in one full rotation.
%\[
%N_{\mbox{\small after}}=\qquad \qquad \qquad \Delta t_{\rm frame}=\]
\[
T_{\mbox{\small after}}=\qquad \qquad \qquad \omega _{\mbox{\small after}}=\]

\end{enumerate}
\answerspace{10mm}
\pagebreak[2]
\textbf{Activity 4: Calculation of Angular Momentum}

(a) Calculate the angular momentum before and after the collision, including UNITS, using the proper equation for each. Calculate the difference between the two results and record it below. 
\vspace{5mm}

\( L_{\mbox{\small before}}= \)  
\vspace{5mm}

\( L_{\mbox{\small after}}= \)
\vspace{5mm}

\( \Delta L= L_{\mbox{\small after}} - L_{\mbox{\small before}} = \)  
\vspace{5mm}

(b) Go around to the other lab groups and get their results for the difference between the angular momenta before and after the collision.
Make a histogram of the results you collect and calculate the average and standard deviation.
For information on making histograms, see \textbf{Appendix \ref{excel}}. For information on calculating the average and standard deviation, see \textbf{Appendix \ref{treatment}}. 
Record the average and standard deviation here. Attach the histogram to this unit.
\vspace{25mm}



(c) What is your expectation for the difference between the initial and final angular momentum?
Do the data from the class support this expectation? 
Use the average and standard deviation for the class to quantitatively answer this question.
\vspace{25mm}

(d) What does the histogram of the class data tell you? Be quantitative in your answer.
\vspace{25mm}

(e) Does the class data indicate a systematic error?  What do you suppose causes this?
\vspace{25mm}

(f) Would the procedure you followed above change if the weight was moving horizontally at a constant velocity when you dropped it? 
If it changed, what would be different?
\vspace{20mm}

