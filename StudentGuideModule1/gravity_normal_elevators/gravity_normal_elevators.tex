\section{Gravity, Normal Force, and Elevators}

\begin{comment}
This lab is based on an activity I've done in class for several years.  This year, I'll try for the first time to write it as a somewhat structured series of written questions.  We'll see how this works!   --Matt Trawick, 9/2019

\end{comment}

\makelabheader %(Space for student name, etc., defined in master.tex or labmanual_formatting_commands.tex)

\begin{wrapfigure}[4]{r}{0.46\textwidth}
    \vspace{-0.3in}
\begin{lab_axis}[lab_noticks_1quad,
	height = {1.5in}, width = {3in},
	ymax=1.1,
	xlabel={Time},
	ylabel={Position},
	xtick={0.1,0.3,0.5,0.7,0.9},
	xticklabels={A,B,C,D,E},
	]
\addplot[domain=0.0 : 0.2] {0.2 };
\addplot[domain=0.2 : 0.4] {5*(x-0.2)^2 + 0.2 };
\addplot[domain=0.4 : 0.6] {2*(x-0.4) + 0.4 };
\addplot[domain=0.6 : 0.8] {-5*(x-0.8)^2 + 1.0 };
\addplot[domain=0.8 : 1.0] {1.0 };
\end{lab_axis}
\end{wrapfigure}

\bigskip
\textbf{Apparatus}

\begin{itemize}[nosep]
\item digital lab scale
\item 1 kg mass
\end{itemize}

\bigskip
\bigskip
\textbf{Activity 1: Some predictions}

%\begin{enumerate}[labparts]
(a) Imagine you are riding in an elevator, and your vertical position as a function of time is given in the graph above.  You begin at rest, accelerate briefly to a constant velocity upwards for a while, then decelerate to a stop at the top floor.  Is there any of the times A through E below when you \textit{feel} heavier than usual?  Which time(s)?
\answerspace{0.4in}


(b) Suppose your normal weight is exactly 100 pounds, and you happen to be standing on a bathroom scale in the elevator as it goes up.  Take a wild guess: During which of the times A through E does the scale read:

\hspace{0.5in} $> 100$ pounds?

\hspace{0.5in} $< 100$ pounds?

\hspace{0.5in} $= 100$ pounds?

\begin{wrapfigure}[8]{r}{0.5\textwidth}
\raggedleft{
\vspace{-0.1in}
\begin{lab_groupplot}{\makegroupverticals[4]{0.2, 0.4, 0.6, 0.8}{0}{1}}
					[lab_noticks_2quads,
	group style={
		group size=1 by 4,
		vertical sep=0.2in,
%	xticklabels at=edge bottom,
		},
	height = {1.5in}, width = {3in},
	xlabel=Time,
	xtick={0.1,0.3,0.5,0.7,0.9},
	xticklabels={A,B,C,D,E},
	]
\nextgroupplot[
	ylabel={Velocity},
	plus_minus_zero_labels,
	]
\nextgroupplot[
	ylabel={Acceleration},
	plus_minus_zero_labels,
	]
\nextgroupplot[
	ylabel={Net force},
	plus_minus_zero_labels,
	]
\nextgroupplot[lab_noticks_1quad,
	ylabel={Normal force},
	xtick={0.1,0.3,0.5,0.7,0.9},
	xticklabels={A,B,C,D,E},
	ylabel_align={0}
	]
\end{lab_groupplot}
} %end of raggedleft
\end{wrapfigure}

\bigskip
(c) Let's think through the last two questions by drawing some additional pictures and graphs.  First, in the space below, draw a free body diagram showing only the forces acting \textit{on you} as you ride the elevator.  Next, fill in the axes to the right with sketches of your velocity, acceleration, $F_{\rm NET}$ and $F_{\rm normal}$ vs. time, being super careful to align your graphs to the position vs. time graph at the top of the page.
\answerspace{1in}

\newpage
(d) Would you care to reconsider either of your answers to parts (a) or (b) now?  Don't erase anything you wrote before; just note any additional thoughts or changes below.
\answerspace{0.8in}


\textbf{Activity 2: An actual experiment, and coming back down}

\begin{enumerate}[labparts]

\item Physics is an experimental science, and you do have at your desposal a scale, a 1-kg mass, and an elevator.  Go for it!  (While you're there, you might consider the return trip back down in the elevator too, as depicted in the graph in the following part.)

\item In the space following the graph below, draw any additional diagrams or graphs you need to explain exactly where you felt lighter than normal, and where you felt heavier than normal.

\hspace{0.2in}
\begin{lab_axis}[lab_noticks_1quad,
	height = {1.5in}, width = {4in},
	ymax=1.1, xmax=2.0,
	xlabel={Time},
	ylabel={Position},
	xtick={0.1,0.3,0.5,0.7,0.9, 1.1, 1.3, 1.5, 1.7},
	xticklabels={A,B,C,D,E,F,G,H,I},
	]
\addplot[domain=0.0 : 0.2] {0.2 };
\addplot[domain=0.2 : 0.4] {5*(x-0.2)^2 + 0.2 };
\addplot[domain=0.4 : 0.6] {2*(x-0.4) + 0.4 };
\addplot[domain=0.6 : 0.8] {-5*(x-0.8)^2 + 1.0 };
\addplot[domain=0.8 : 1.0] {1.0 };
\addplot[domain=1.0 : 1.2] {-5*(x-1.0)^2 + 1.0 };
\addplot[domain=1.2 : 1.4] {-2*(x-1.4) + 0.4 };
\addplot[domain=1.4 : 1.6] {5*(x-1.6)^2 + 0.2 };
\addplot[domain=1.6 : 1.8] {0.2 };
\end{lab_axis}

\vfill

\item Bottom line:  If your normal weight was exactly 100 pounds, during which of the times A through I does the scale read:

\hspace{0.5in} $> 100$ pounds?

\hspace{0.5in} $< 100$ pounds?

\hspace{0.5in} $= 100$ pounds?

\end{enumerate}


