\usepackage{endnotes}

% The following code is for handling endnotes, used for notes for instructors.
% see https://tex.stackexchange.com/questions/84914/chapter-numbers-in-endnotes

\renewcommand\notesname{Instructor Notes}

% Note that it redefines \section AGAIN
\makeatletter
\renewcommand\enoteheading{%
  \setcounter{secnumdepth}{-2}
%  \chapter*{\notesname}
  \section*{\notesname}
  \mbox{}\par\vskip-\baselineskip
%  \let\@afterindentfalse\@afterindenttrue
}
\makeatother

\let\latexsection\section

\RenewDocumentCommand{\section}{som}{%
  \IfBooleanTF{#1}
    {\latexsection*{#3}}
    {\IfNoValueTF{#2}
       {\latexsection{#3}}
       {\latexsection[#2]{#3}}%
     \addtoendnotes{%
%\unexpanded{\enotedivision{\subsection}{#3}}}
     \noexpand\enotedivision{\noexpand\subsection}
%         {\sectionname\ \thesection. \unexpanded{#3}}}%
         {\headersupplementmark\ \thesection: \unexpanded{#3}}}%   }%
	}
}
\makeatletter
\def\enotedivision#1#2{\@ifnextchar\enotedivision{}{#1{#2}}}
\makeatother

\newcommand{\startendnotes}{
	\addtoendnotes{\unexpanded{\enotedivision{}{}}}

	\begingroup
	\parindent 0pt
	\parskip 2ex
	\def\enotesize{\normalsize}
	\def\makeenmark{}%\theenmark}
	\setlength\parindent{0pt}
\fancyhead[LO,RE]{\slshape \rightmark} 
\fancyhead[LE,RO]{\slshape \MakeUppercase{\notesname}}

%	\renewcommand{\footnote}[1]{foo foo foo}
%	\def\footnote{}
	\theendnotes
	\endgroup

}
% End of code for endnotes
