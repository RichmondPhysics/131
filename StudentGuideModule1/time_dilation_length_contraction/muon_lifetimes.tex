\section{Muon Lifetimes}
%By Matt Trawick.

\instructornote{%
By Matt Trawick, 2019.  Time: ?

This lab was initially included as part of the ``time dilation and length contraction'' lab.  I split it off in spring 2021.

The lab is an exercise involving the lifetime of muons created in the atmosphere, a classic piece of experimental evidence for time dilation.  It considers the journey through the atmosphere from the reference frames of the Earth and the muons, giving practice in both time dilation and length contraction.  Answers: (a) $1.43 \times 10^{-5}$~sec (b)~0.15\% (d)~$5.3 \times 10^{-6}$~sec (e) 9.1\% 
}

\makelabheader %(Space for student name, etc., defined in master.tex)

\bigskip

\textbf{Background} 

Muons are rare subatomic particles which are not usually found in ordinary matter on Earth.  Muons can be created when protons or atomic nuclei, traveling at high speeds from the sun or other stars, strike the Earth's upper atmosphere and collide with molecules in the air.  As a result of these collisions, there are always muons raining down from the upper atmosphere towards the Earth's surface.

But muons don't last long.  At rest in a laboratory, muons have an average lifetime of only 2.2~$\mu$s, spontaneously decaying into (usually) an electron and two neutrinos.  Given an initial number of muons $N_0$ at time $t=0$, the number of muons $N$ at any later time $t$ is given by
\begin{equation}
N=N_0 e^{-t/\tau},
\end{equation}
where $\tau = 2.2~\mu$s is the muons' average lifetime.  Note that this equation can also be written as $N=N_0 e^{-\lambda t}$, where $\lambda$ is called the ``decay constant,'' and is simply $1/\tau$.  

\providecommand{\halflife}{t_{\rm \nicefrac{1\mkern-1mu}{\mkern-1mu2}}}

The average lifetime $\tau$ of a muon is slightly different from the ``half life'' of a muon.  The half life $\halflife$ would be the time it takes for the number of muons to fall to 1/2 of the original number. The average lifetime $\tau$ used in this lab is the time it takes for the number of muons to fall to $1/e$ of the original number.

\bigskip

\textbf{Activity 1: Working through an example of muon lifetimes} 

\begin{enumerate}[labparts]
\item   
Suppose that muons are created in the Earth's atmosphere at a height of 4~km above the surface.  The muons stream towards the ground at a speed of $0.93c$.
In the reference frame of the Earth, how long does it take these muons to reach the ground?
(This is a straightforward question using physics you likely learned in high school.)
\answerspace{0.9in}

\item What percent of the original muons should be remaining after this time, according to Equation (1)? 
\answerspace{0.7in}

\item The time $\Delta t$ that you calculated in (a) is the time between two specific events: the muon being created and the muon hitting the Earth's surface.  What reference frame measures the \textit{proper time} $\Delta t_0$ between these two events?  Is it the reference frame of a scientist on Earth (called the \textit{Earth frame} or \textit{lab frame}), or the reference frame of the muon?
\answerspace{0.7in}

\item What is the time of the journey from the upper atmosphere to the ground in the reference frame of the muon, traveling at speed $0.93c$?
\answerspace{0.8in}

\item What percent of the original muons should be remaining after this time $\Delta t_0$, in the reference frame of the muons?
\answerspace{0.6in}

\item Whatever causes a muon to decay takes place entirely within the muon itself.  (That's why it's called ``spontaneous'' decay; it's not caused by anything external to the muon.)  
Is the time $t$ that governs the average lifetime of a muon the time as measured in the Earth's reference frame, or the muon's reference frame?  
\answerspace{0.5in}

\item If you measure the fraction of muons that \textit{actually} reach the ground, what fraction will you observe?
(This observation was first done in the 1920s, and was an early experimental confirmation of Einstein's theory of relativity.)
\answerspace{0.5in}

\item One more thing.  In part (d), you probably calculated the muons' $\Delta t_0$ using the idea of time dilation.  Now, try doing it again using the idea of length contraction.  In the reference frame of the muon, what is the distance between the upper atmosphere and the ground?  (\textit{Hint: Who measures the proper length $L_0$ of the atmosphere?  Is it a scientist on Earth, or the muons?})
\answerspace{0.7in}

\item How long does it take the muon to travel that distance?
\answerspace{0.6in}

\end{enumerate}
\bigskip

\textbf{Activity 2: An ``effective average lifetime''?}
\begin{enumerate}[labparts]
\item  
Anna and Bob are doing a lab about muon lifetimes.  Bob looks at their calculations and says, ``Interesting! It looks like the speed of a particle effectively changes its average lifetime, at least as people in the lab would measure it.''  Is this statement reasonable?
\answerspace{0.6in}

\item Does moving at a faster speed make a particle's ``effective'' average lifetime \textit{longer} or \textit{shorter} than if it were standing still?
\answerspace{0.6in}

\item If you know the average lifetime of a particle when it is at rest, how would you calculate its ``effective'' average lifetime (according to someone in the lab frame) when the particle moves at speed $v$?
\answerspace{0.9in}

\item The most recently discovered element is Tennessine (named after the state of Tennessee) which has an atomic number of 117.  It is never found naturally, because it is highly unstable; its most stable isotope, $^{294}$Ts, has a half life of $\halflife = 51$~ms.  What would be the ``effective'' average lifetime $\tau$ of a beam of $^{294}$Ts nuclei with speed $v=0.9c$, as measured in the lab frame?
\answerspace{0.9in}
\vspace{\fill}

\hspace{\fill}\textit{Answer to part (d): 169~ms}
\end{enumerate}


