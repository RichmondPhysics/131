
\section{Conservation of Mechanical Energy\footnote{
1990-93 Dept. of Physics and Astronomy, Dickinson College. Supported by FIPSE
(U.S. Dept. of Ed.) and NSF. Portions of this material may have been modified
locally and may not have been classroom tested at Dickinson College.
}}

\makelabheader %(Space for student name, etc., defined in master.tex or labmanual_formatting_commands.tex)

\textbf{Objectives }

\begin{itemize}
\item To understand the concept of potential energy. 
\item To investigate the conditions under which mechanical energy is conserved.
\end{itemize}
\textbf{Overview }

The last unit on work and energy culminated with a mathematical proof of the
work-kinetic energy theorem for a mass falling under the influence of the 
force of gravity. We found that when a mass starts from rest and falls a 
distance $y$, its final velocity can be related to $y$ by the familiar 
kinematic equation
\[
v_{f}^{2}=v_{i}^{2}+2gy\qquad \mbox{or}
\qquad gy=\frac{1}{2}\left( v_{f}^{2}-v_{i}^{2}\right) \qquad [Eq.\: 1]\]


where \( v_{f} \) is the final velocity and \( v_{i} \) is the initial velocity
of the mass.

We believe this equation is valid because: (1) you have derived the kinematic
equations mathematically using the definitions of velocity and constant 
acceleration, and (2) you have verified experimentally that masses fall at a 
constant acceleration.
We then asked whether the transformation of the mass from a speed \( v_{i} \)
to a speed \( v_{f} \) is related to the work done on the mass by the force
of gravity as it falls.

The answer is mathematically simple. Since \( F_{g}=mg \), the work done
on the falling object by the force of gravity is given by
\[
W_{g}=F_{g}y=mgy\qquad [Eq.\: 2]\]


But according to Equation 1, \(gy = \frac{1}{2} v_{f}^{2} - \frac{1}{2} 
v_{i}^{2} \),
so we can re-write Equation 2 as
\[
W_{g}=mgy=\frac{1}{2}mv_{f}^{2}-\frac{1}{2}mv_{i}^{2}\qquad [Eq.\: 3]\]


The \(\frac{1}{2}mv_{f}^{2} \) is a measure of the motion resulting from the fall.
If we define it as the energy of motion, or, more succinctly, the kinetic 
energy, we can define a work-kinetic energy theorem for falling objects:
\[
W=\Delta K\qquad [Eq.\: 4]\]


or, the work done on a falling object by the earth is equal to the change in
its kinetic energy as calculated by the difference between the final and initial
kinetic energies.

If external work is done on the mass to raise it through a height $y$ (a fancy
phrase meaning ``if some one picks up the mass''), it now has
the potential to fall back through the distance $y$, gaining kinetic energy as
it falls. Aha! Suppose we define \textit{potential energy} to be \textit{the
amount of external work, \( W_{ext} \), needed to move a mass at constant velocity
through a distance $y$ against the force of gravity.} Since this amount of work
is positive while the work done by the gravitational force has the same magnitude
but is negative, this definition can be expressed mathematically as
\[
U=W_{ext}=mgy\qquad [Eq.\: 5]\]


Note that when the potential energy is a maximum, the falling mass has no kinetic
energy but it has a maximum potential energy. As it falls, the potential energy
becomes smaller and smaller as the kinetic energy increases. The kinetic and
potential energy are considered to be two different forms of mechanical energy.
What about the total mechanical energy, consisting of the sum of these two energies?
Is the total mechanical energy constant during the time the object falls? If
it is, we might be able to hypothesize a law of conservation of mechanical energy
as follows: \textit{In some systems, the sum, E, of the kinetic and potential
energy is a constant.} This hypothesis can be summarized mathematically by the
following statement.
\[
E=K+U=\mbox{constant}\qquad [Eq.\: 6]\]


The idea of mechanical energy conservation raises a number of questions. Does
it hold quantitatively for falling masses? How about for masses experiencing
other forces, like those exerted by a spring? Can we develop an equivalent definition
of potential energy for the mass-spring system and other systems and re-introduce
the hypothesis of conservation of mechanical energy for those systems? Is mechanical
energy conserved for masses experiencing frictional forces, like those encountered
in sliding?

In this unit, you will explore whether or not the mechanical energy conservation
hypothesis is valid for a falling mass.

\textbf{Activity 1: Mechanical Energy for a Falling Mass} 

Suppose a ball of mass $m$ is dropped from a height $h$ above the ground.

(a) Where is $U$ a maximum? A minimum?
\vspace{20mm}

(b) Where is $K$ a maximum? A minimum?
\vspace{20mm}

(c) If mechanical energy is conserved what can you say about the sum of $K+ U$ 
for any point along the path of a falling mass?
\vspace{20mm}


\textbf{Mechanical Energy Conservation }

How do people in different reference frames near the surface of the earth view
the same event with regard to mechanical energy associated with a mass and its
conservation? Suppose the president of your college drops a 2.0-kg 
water balloon
from the second floor of the administration building (10.0 meters above the
ground). The president takes the origin of his or her vertical axis to be even
with the level of the second floor. A student standing on the ground below considers
the origin of his coordinate system to be at ground level. Have a discussion
with your classmates and try your hand at answering the questions below.

\textbf{Activity 2: Mechanical Energy and Coordinate Systems} 

(a) What is the value of the potential energy of the balloon before and after
it is dropped according to the president? According to the student? Show your
calculations and don't forget to include units!

The president's perspective is that $y = 0.0$ m at $t = 0$ s and that 
$y = -10.0$ m
when the balloon hits the student): 
\vspace{5mm}

\( U_{i} \) = 
\vspace{5mm}

\( U_{f} \) =
\vspace{5mm}

The student's perspective is that $y = 10.0$ m at $t = 0$ s and that 
$y = 0.0$ m when
the balloon hits: 
\vspace{5mm}

\( U_{i} \) = 
\vspace{5mm}

\( U_{f} \) =
\vspace{5mm}

Note: If you get the same potential energy value for the student and the president,
you are on the wrong track!

(b) What is the value of the kinetic energy of the balloon before and after
it is dropped according to the president? According to the student? Show your
calculations. Hint: Use a kinematic equation to find the velocity of the balloon
at ground level.

President's perspective: 
\vspace{5mm}

\( K_{i} \) = 
\vspace{5mm}

\( K_{f} \) =

Student's perspective: 
\vspace{5mm}

\( K_{i} \) = 
\vspace{5mm}

\( K_{f} \) =
\vspace{5mm}

Note: If you get the same values for both the student and the president for
values of the kinetic energies you are on the right track!

(c) What is the value of the total mechanical energy of the balloon before and
after it is dropped according to the president? According to the student? Show
your calculations. Note: If you get the same values for both the student and
the president for the total energies you are on the wrong track!!!!!

President's perspective: 
\vspace{5mm}

\( E_{i} \) = 
\vspace{5mm}

\( E_{f} \) =
\vspace{5mm}

Student's perspective: 
\vspace{5mm}

\( E_{i} \) = 
\vspace{5mm}

\( E_{f} \) =
\vspace{5mm}

(d) Why don't the two observers calculate the same values for the mechanical
energy of the water balloon? 
\vspace{15mm}

(e) Why do the two observers agree that mechanical energy is conserved? 


