% This file loads all packages and makes all definitions used for making various graphs,
% using Tikz and pgfplots.  It was developed in July-August 2019 by Matt Trawick.
% The goal is to pre-define several basic types of graphs used in our manuals (plain axes
% with qualitative graphs, or empty grids with scales, etc.) that could be easily used and
% tweaked throughout the manual, for stylistic uniformity.
% These could gradually (or not so gradually) replace the hodge-podge of .eps files
% generated from difference sources that are currently used in the manuals.
\usepackage{pgfplots}
\usepackage{sansmath} 
\usepgfplotslibrary{groupplots}
\usetikzlibrary{pgfplots.groupplots}
\usetikzlibrary{math}
\usepackage{xparse} %used for \NewDocumentCommand
% xparse is actually used elsewhere as well, so doesn't need to be called here for the main lab manual.
% I'm keeping the line in for when this file is called by another .tex file, for instance if you're just testing
% how things work.

\pgfplotsset{compat=1.16} %Version 1.16 is the current version, as of July 2019

%-------------------------------------------------------------------------
% Below, set several style choices used throughout the manual.
\tikzset{every picture/.style={font=\sffamily}}

\pgfplotsset{
  tick label style = {font=\sansmath\sffamily},
%  every axis label = {font=\sansmath\sffamily},
%  legend style = {font=\sffamily},
%  label style = {font=\sffamily}
	samples = 200,
	height = {2in}, width = {2in},
}

% This pre-defines a "cycle" of colors and styles that can be used for successive plots added to an axis.
% In this case the "cycle" has only one entry, so all plots look the same.
\pgfplotscreateplotcyclelist{lab_solid}{
	{black}
}

%---------------------------------------------------------------------------
% This part predefines several named styles ("lab_noticks_1quad" for a single-quadrant set of blank axes, etc.)
% that can be called for various graphs used throughout the manual.
% Each option within the defined style can be overridden individually as needed, or
% additional options added. For instance,
% 		\begin{lab_axis}[lab_noticks_2quads,
% 			ymin = -0.5,
% 			....
%
% These "styles" are really just lists of options, and can be combined if needed, as in 
% 		\begin{lab_axis}[lab_noticks_2quads,
%			plus_minus_zero_labels
% 			....

% single quadrant:
\pgfplotsset{lab_noticks_1quad/.style={
	cycle list name = lab_solid,
	every axis plot/.append style={very thick},
	axis x line={left, x axis line style={->}}, % solid line with no arrow.
	axis y line={left, y axis line style={->}},
	xmin=0, xmax=1,
	ymin=0, ymax=1, %Although there are no ticks, this seems to be needed to make the ticks disappear.
	tick style={xtick=\empty,ytick=\empty}
}}

% two quadrants (I and IV):
\pgfplotsset{lab_noticks_2quads/.style={
	cycle list name = lab_solid,
	every axis plot/.append style={very thick},
	axis x line= {middle, x axis line style={->}}, 
	axis y line= {left,  y axis line style={<->}},
      xlabel style={at=(current axis.right of origin), anchor=west},
	xmin=0, xmax=1,
	ymin=-1, ymax=1, 
	tick style={xtick=\empty,ytick=\empty},
}}

% two quadrants (I and IV), but with labels like for a mathematical function:
\pgfplotsset{lab_noticks_2quads_algebraic/.style={
	cycle list name = lab_solid,
	every axis plot/.append style={very thick},
	axis x line= {middle, x axis line style={->}}, 
	axis y line= {middle, y axis line style={<->}},
	xlabel style={at=(current axis.right of origin), anchor=north east},
	ylabel style={at=(ticklabel* cs:1.0), anchor=north east},
	xmin=0, xmax=1,
	ymin=-1, ymax=1, 
	tick style={xtick=\empty,ytick=\empty},
}}

%four quadrants:
\pgfplotsset{lab_noticks_4quads/.style={
	cycle list name = lab_solid,
	every axis plot/.append style={very thick},
	axis x line= {middle, x axis line style={<->}}, 
	axis y line= {middle, y axis line style={<->}},
	xlabel style={at=(current axis.right of origin), anchor=north east},
	ylabel style={at=(ticklabel* cs:1.0), anchor=north east},
	xmin=-1, xmax=1,
	ymin=-1, ymax=1, 
	tick style={xtick=\empty,ytick=\empty},
}}

\pgfplotsset{lab_grid/.style={
	cycle list name = lab_solid,
	every axis plot/.append style={very thick},
%	tick style={grid=both,major tick length=0pt},
	axis line style={draw=none},
	tick style={
		grid=both, 
		major tick length=0pt,
		minor tick length=0pt},
}}


% Use this to add a darker horizontal line at y=0 to lab_grid style graphs.
\pgfplotsset{y0_line/.style={
	extra y ticks={0},
	extra tick style={
		major grid style=black,
		},
}}


% In a groupplot, the y labels can be misaligned if one graph has "longer" tick lables, such as ones that include
% a negative sign.  This command bumps the y label of the current to the left, to match the alignment of the
% y label of the other graph.  To tell it how much to move it over, you need to give it the longest tick label text 
% of the other graph as an argument.  Example: 
% 		ylabel_align={-10}
% Note that it works by actually creating an "extra y tick" on the current graph, but writing the label as
% phantom text that doesn't show up.  For this reason, it will not work with y0_line, because that also works
% by defining an "extra y tick" at zero.
\pgfplotsset{ylabel_align/.style={
	extra y ticks={\pgfkeysvalueof{/pgfplots/xmin}},
	extra y tick labels={\phantom{$#1$}}, 
}}

% Use this to add y axis labels +, -, and 0 to graphs, presumably to lab_noticks_2quads style graphs.
\pgfplotsset{plus_minus_zero_labels/.style={
	extra y tick style={major tick length=0pt},
	extra y ticks = {-0.8, 0, 0.8},
	extra y tick labels = {\scriptsize{$-$},0,\scriptsize{$+$}},
}}


% The commmand below makes vertical alignment marks through all graphs in a groupplot.
% (Restricted to only a single column, but multiple rows.  The number of rows can be "1" though.)
%
% How to call:
% 		\makegroupverticlas[rows]{x1,x2,x3...}{xmin}{xmax}
% rows is assumed to be 2 if not specified.  xmin and xmax are the min and max values of the axis.
% Oh, and it won't work if you define a name for the group; it must remain the default "group".
%
% This was a giant pain to get to work.  
% The problem seems to be that INSIDE the groupplot environment, tikz knows about xmin and xmax
% (since it can get it from \pgfkeysvalueof{/pgfplots/xmin} ) but does not know about the top and bottom
% of the whole group; only of a single axis.
% But OUTSIDE the grouplot environment, tikz knows about the top and bottom of the group
% (from group c1r1.north west, for instance), but does not know about the scale of the x axis.
% Or at least I couldn't figure out how to get it to know about the x axis.  Maybe someone else can.
% If so, it would be a nice improvement to NOT have to pass it xmin and xmax, or even the number of
% rows, for that matter. 
\NewDocumentCommand \makegroupverticals {o m m m} {
	\coordinate (TL) at (group c1r1.north west); % top left
      \IfNoValueTF  {#2} {%If no lastrow given, assume two rows
		\coordinate (BR) at (group c1r3.south east); % bottom right
		}
	{ % else use the value for the number of the last row
		\coordinate (BR) at (group c1r#1.south east); % bottom right
		}
	\newdimen\xleft
	\pgfextractx\xleft{\pgfpointanchor{TL}{center}}
	\newdimen\xright
	\pgfextractx\xright{\pgfpointanchor{BR}{center}}

	\newdimen\xx
	\foreach \x in {#2} { 
		\tikzmath{\xx = \xleft + (\xright - \xleft) * ((\x - #3) / (#4 - #3))  ;}
		\draw[black, thick, dashed] (\xx, |- TL) -- (\xx, |- BR);
		}
}

% Here's an older version of makegroupverticals.
% It does not work, because the x axis coordinates in \x are often wrong,
% Sometimes by a huge amount.  I suspect there's a way to make this work,
% with a small fix, so I'll keep it here.
\NewDocumentCommand \oldmakegroupverticals {o m} {
	\coordinate (toppoint) at  (group c1r1.north);
      \IfNoValueTF  {#2} {%If no lastrow given, assume two rows
		\coordinate (botpoint) at  (group c1r2.south);
		}
	{ % else use the value for the number of the last row
		\coordinate (botpoint) at  (group c1r#1.south);
	}
	\foreach \x in {#2} 
		\draw [black, very thick, dashed] (\x, |- toppoint) -- (\x, |- botpoint);
}


%------------------------------------------------------------------------------------------------------------------
% Finally, this next section defines two (currently) environments, simply for convenience, to make
% these graphs less cumbersome to call.  You can still make graphs by using 
% 		\begin{tikzpicture}
% 		\begin{axis}
%		 .....
% if you want.


% The lab_axis environment, defined below, is basically a thin wrapper around the "tikzpicture" and "axis" 
% environments, combined.  When called in the starred version ("*", see below) it also centers the graph.
% Note that it also puts small vertical spaces above and below the graph.  If this is called multiple times on 
% one line, the vertical spaces will build up and be placed after the graphs, requiring a 
%		\vspace{-2em} 
% (or whatever length you need) to undo.
\NewDocumentEnvironment{lab_axis}{s}{
	\IfBooleanTF #1
		{ \begin{center} } %if starred
		{} %if not starred, do nothing
	\vspace{0em} % Because our lab manual skips lines between paragraphs, we don't need extra space 
	\begin{tikzpicture} 
	\begin{axis}
	}
	{\end{axis} 
	\end{tikzpicture}
	\vspace{0em}
	\IfBooleanTF #1
		{\end{center} } %if starred
		{} %if not starred, do nothing
	 }
% Next four lines allow either 
%	\begin{lab_axis*}... \end{lab_axis*} 
% or 
% 	\begin{lab_axis}*... \end{lab_axis}
\ExplSyntaxOn
\cs_new:cpn {lab_axis*} {\lab_axis*}
\cs_new_eq:cN {endlab_axis*} \endlab_axis
\ExplSyntaxOff

% This new environment is a thin wrapper around the "tikzpicture" and "groupplot" environments, combined.
% When called in the starred version, it is centered as well.  (Here, the "*" goes AFTER the "}", as in
% 		\begin{lab_groupplot}*{}blah blah blah.
% The empty argument above is where you put any line or lines that go after the \end{groupplot} but
% before the \end{tikzpicture}, which would typically be drawing vertical alignment marks
% using \makegroupverticals, or adding other annotations.
\NewDocumentEnvironment{lab_groupplot}{s m}{
	\IfBooleanTF #1
		{ \begin{center} } %if starred
		{} %if not starred, do nothing
	\vspace{1em}
	\begin{tikzpicture}
	\begin{groupplot}
	}
	{\end{groupplot}
	#2
	\end{tikzpicture}
	\vspace{1em}
	\IfBooleanTF #1
		{\end{center} } %if starred
		{} %if not starred, do nothing
	 }

