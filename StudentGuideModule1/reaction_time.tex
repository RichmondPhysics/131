
\section{Human Reaction Time}

\makelabheader %(Space for student name, etc., defined in master.tex or labmanual_formatting_commands.tex)

{\noindent \bf Objectives:} \begin{list}{$\bullet$}{\itemsep0pt \parsep0pt}

\item Application of the free-fall concept \item Appreciation of the magnitude of reaction time

\end{list}

{\noindent \bf Introduction}

{\noindent Since you now know the magnitude of gravitational acceleration, you can determine your reaction time from the distance a meter stick drops before you catch it.} 
\vspace{0.2in}

{\noindent \bf Apparatus:} \begin{list}{$\bullet$}{\itemsep0pt \parsep0pt}

\item meter stick \item stop watch

\end{list}

{\noindent \bf Activity:} [Note: everyone should determine their individual reaction time.]

\begin{enumerate}

\item With a lab partner holding a meter stick vertically from the 0.0 end, hold your thumb and index finger close to, but not touching, opposite sides of the stick at the 50.0-cm mark.

\item The holder should release the stick without warning and you should try to grasp it as quickly as possible. Record the position at which you caught the stick.

\item Repeat the exercise five times. Then calculate the average position and the characteristic displacement, s, of the stick before you can catch it.

\begin{center} \begin{tabular}{|c|c|c|c|c|c|c|} \hline $P_1$ & $P_2$ & $P_3$ & $P_4$ & $P_5$ & Ave. $P$ & $s = 50.0 - P$ \\ \hline \hline & & & & & & \\ \hline \end{tabular} \end{center}

\item Calculate your reaction time, $t$. (show calculations)

\bigskip \bigskip

\center {Reaction time $\equiv$ $t =$ \rule{2in}{0.2pt}} \end{enumerate}

\medskip

{\noindent \bf Questions:}

\begin{enumerate}
\item How many feet would a car traveling 55 mi./hr travel during your reaction time?
(show calculations) \vspace{10mm}

\item Could you catch a dropped dollar bill (15.5 cm long) if your fingers were initially
at the a) center, b) bottom? Use the experimental data just taken to justify
your answers.
\end{enumerate}
