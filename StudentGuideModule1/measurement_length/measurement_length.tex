
\section{Measurement of Length, Mass, Volume, and Density}

\makelabheader %(Space for student name, etc., defined in master.tex or labmanual_formatting_commands.tex)

\vspace{0.2in}
{\noindent \bf Objectives:} \begin{itemize}
%{$\bullet$}{\itemsep0pt \parsep0pt}

\item Learn to measure length with the vernier caliper and mass with the platform balance 
\item Apply knowledge of units and significant figures 
\item Understand the dependence of mass and density on dimensions

\end{itemize}

{\noindent \bf Apparatus:} \begin{itemize}
%{$\bullet$}{\itemsep0pt \parsep0pt}

\item vernier caliper 
\item platform balance 
\item set of wooden disks

\end{itemize}

{\noindent \bf Activity:} \begin{enumerate}

\item Find the dimensions in centimeters of each of the disks using the vernier caliper (see Appendix \ref{instrumentation}). Use the averages of three trials for each dimension (diameter, $D$; width, $W$) in the calculations of the volumes ($V = \pi r^2 W$).
\answerspace{0.5in}

\item Find the mass, $M$, of each disk using the laboratory balance.
\answerspace{0.5in}

\item Calculate the density, $\rho$.
\answerspace{0.5in}

\begin{center} \begin{tabular}{|c|c|c|c|c|c|c|c|c|c|c|c|} \hline \multicolumn{1}{|c||}{} & $D_1$ & $D_2$ & $D_3$ & $W_1$ & $W_2$ & $W_3$ & $D$ & $W$ & $V$ & $M$ & $\rho$\\ \multicolumn{1}{|c||}{disk} & (cm) & (cm) & (cm) & (cm) & (cm) & (cm) & (cm) & (cm) & (cc) & (g) & (g/cc) \\ \hline \hline \multicolumn{1}{|c||}{1} & & & & & & & & & & & \\ \hline \multicolumn{1}{|c||}{2} & & & & & & & & & & & \\ \hline \multicolumn{1}{|c||}{3} & & & & & & & & & & & \\ \hline \multicolumn{1}{|c||}{4} & & & & & & & & & & & \\ \hline \multicolumn{1}{|c||}{5} & & & & & & & & & & & \\ \hline \end{tabular} \end{center}

\item Graph mass versus radius and mass versus radius squared.
\answerspace{0.5in}

\end{enumerate}

\pagebreak[2]
%\newpage

{\noindent \bf Questions:}

1. How does the density depend on the size of a disk? 
\vspace{20mm}

2. What is the nature of the relationship between mass and radius? What is the
dependency? 
\vspace{20mm}

3. What is the smallest part of a centimeter that can be read or estimated with
a meter stick? With a vernier caliper? Which reading is more reliable? Explain.
\vspace{20mm}

4. When determining the volume of a disk, which dimension, diameter or width,
should be measured more carefully? Explain. 
\vspace{20mm}

5. What is the volume of the largest disk in cubic millimeters? In liters? What
is its mass in kilograms?

