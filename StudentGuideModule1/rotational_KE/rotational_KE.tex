
\section*{Rotational Kinetic Energy\footnote{
2018 Department of Physics, Univeristy of Richmond. C.C. Helms.
}}

\makelabheader %(Space for student name, etc., defined in master.tex or labmanual_formatting_commands.tex)

\textbf{Objective} 

To determine the change in kinetic energy of a rotating disk and falling mass system and compare with the change in potential energy of the system. 


\textbf{Apparatus}

\begin{itemize}
\item A Rotating Disk System and string
\item A hanging mass of 100 g  
\item Table clamp and rotary motion sensor
\item meter stick
\item scale
\item calipers
\end{itemize}
\textbf{Overview }

Conservation of energy states that the change in energy of an isolated system equals zero. Therefore, for our isolated system any change in potential energy should be equal to the change in kinetic and thermal energy. By minimizing friction we will observer the conversion of potential energy to kinetic energy.  We will compare a system with purely translational motion to one with both translational and rotational motion. 

Since many objects undergo rotational motion it is useful to be able to describe
their motions mathematically. The study of rotational motion is also very useful
in obtaining a deeper understanding of the nature of linear and parabolic motion.

In this experiment kinetic energy arises from allowing a mass to fall while attached to a disk, which is free to spin. If the mass falls a distance (h) the mass loses gravitational potential energy and gains kinetic energy.   


\textbf{Kinetic Energy of the system} 

(a) What objects begin to move once the weight begins to fall?
\answerspace{7mm}

(b) What type of kinetic energy does each object have? Write the equation for the kinetic energy for each object.
\answerspace{7mm}

\textbf{Moment of Inertia} 

We have used the definition of moment of inertia, I, to determine a theoretical equation for the moment of inertia of a uniform disk.  The equation was given by
\[
I=\frac{1}{2}MR^{2}.\]

Does this equation adequately describe the moment of inertia of the rotating disks? 

\answerspace{3mm} 

(a) Calculate the moment of inertia of the disk on the rotational motion apparatus (which radius do you use?) Ignore the small hole in the middle in your calculations (assume the disk is uniform)
 \answerspace{15mm}
 
(b) Estimate the moment of inertia for the disk on the front of the rotary motions sensor? We will ignor the motion of the rotary motion sensor in this lab. Is it reasonable to assume the rotational kinetic energy is dominated by the disk on the rotational motion apparatus and ignor the motion of the of the rotary motion sensor? Why or why not?
 \answerspace{15mm}


\textbf{Activity 1: Theoretical Calculations}

(a) Determine the height (h) through which your mass will fall.
\answerspace{5mm}

(b) Calculate the final velocity of a 100 g mass dropped from from this height (assuming a simple system with no rotation)
\answerspace{10mm} 

(c) Using conservation of energy, write an equation (in terms of variables such as I, \( \omega \), R, r, m...) for the change in potential and kinetic energy of our experimental system.
\answerspace{10mm}

(d) Rewrite the equation for energy substituing in the linear velocity of the string and mass for the angular velocity of the disk. Below the equation indicate which radius each r or R in the equation is refering to and which mass each m or M is refering to. 
\answerspace{20mm}


(e) We will measure the angular velocity of the rotary motion sensor. How does the angular velocity of the rotary motion sensor \( \omega_{rm}\) relate to the  
linear velocity of the string \( v_{s}\)? What radius do you use? 

\( v_{s}\) =
\answerspace{5mm}

(f) Record the radius of the rotary motion sensor.

\( r_{ms}\) =
\answerspace{3mm}

(g) Calculate the final velocity of the mass in your system, assuming a 100 g mass falling through the height your recorded above.
\answerspace{20mm}

\newpage

\textbf{Activity 2: Measuring motion}  

(a) Open the rotational KE capstone file available on blackboard 

(b) Place the 100 g mass on the string and place the string over the middle (medium) pulley grove on the rotary motion sensor. 

(c) Record the starting position of the mass.
\answerspace{2mm}

\( h_{i} \)=
\answerspace{2mm}

(d) Click run and release the mass

(e) Analyze your data. You may need to repeat c and d multiple times to obtain good interpretable data.
What was the maximum angular velocity of the rotatory motion sensor?
\answerspace{3mm} 

What is the linear velocity of the string and mass \(v_{s} \) (see part e above)?
\answerspace{3mm} 

(f) Repeat 2 additional times 


(g) How does your theoretical velocity (Activity 1: g) compare with your experimental velocity?  Suggest some reasons for any differences.

\newpage
