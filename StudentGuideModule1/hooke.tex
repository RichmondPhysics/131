
\section{Hooke's Law}

Name \rule{2.0in}{0.1pt}\hfill{}Section \rule{1.0in}{0.1pt}\hfill{}Date \rule{1.0in}{0.1pt}

{\noindent \bf Objectives:} \begin{list}{$\bullet$}{\itemsep0pt \parsep0pt}

\item To explore the nature of elastic deformation and restoring forces

\end{list}

\noindent {\bf Introduction} 

There is no such thing as a perfectly rigid body. The stiffest of metal bars can be twisted, bent, stretched, and compressed. Delicate measurements show that even small forces cause these distortions. Under certain circumstances (typically, when the forces are not too large), a body deformed by forces acting upon it will return to its original size and shape when the forces are removed, a capacity known as elasticity. Permanent distortion from large forces is referred to as plastic deformation. In this lab, you will stay within the elastic limit. \\

{\noindent \bf Apparatus:} \begin{list}{$\bullet$}{\itemsep0pt \parsep0pt}

\item two springs and supports \item collection of masses \item 2-meter stick

\end{list}

{\noindent \bf Activity:} \begin{enumerate}

%\item  Record the mass of the two springs.

%\begin{displaymath} m_1 = \hskip50pt  m_2 = \end{displaymath}

\item  Suspend one of the springs from the support. Using the meter stick, observe the position of the lower end of the spring and record the value in the table below.

\item  Hang 100 grams from the lower end of the spring and again record the position of this end.

\item  Repeat 2 with loads of 200, 300, 400, and 500 grams hung from the spring.

\item  Repeat 2 and 3 with the second spring.

\begin{center} \begin{tabular}{||c|c|c|c||c|c|c||} \hline \hline Mass suspended & Force on & Position & Elongation & Mass suspended & Position & Elongation \\ from spring 1 & spring 1 & reading & & from spring 2 & reading & \\ (g) & (N) & (m) & (m) & (g) & (m) & (m) \\ \hline \hline 0 & 0 & & 0 & 0 & & 0 \\ \hline 100 & & & & 100 & & \\ \hline 200 & & & & 200 & & \\ \hline 300 & & & & 300 & & \\ \hline 400 & & & & 400 & & \\ \hline 500 & & & & 500 & & \\ \hline \hline \end{tabular} \end{center}

\item Determine the elongation produced by each load.

\item  Plot a curve using the values of the elongation as the abscissas (x values) and the forces due to the corresponding loads as ordinates (y values) for each spring. Make sure you use compatible units. Write the equation for each curve in the space below.

\vskip70pt

\end{enumerate}

\pagebreak

\textbf{Questions:}

\begin{enumerate}
\item What do your curves show about the dependence of each spring's elongation upon the applied force?\vspace{30mm}

\item List the proportionality constant (including proper units) for each spring in the space below.\vspace{30mm}

\item The slope of each line (the proportionality constant) is known as the 
force constant, $k$. Use the LINEST function in $Excel$ (see \textbf{Appendix C:
 Excel}) to determine the slope of the line and the uncertainty in the slope. 
Write the force constant as $k$ = $slope$ \( \pm \ \Delta  slope\).  Be sure to 
include the proper units.
\end{enumerate}
