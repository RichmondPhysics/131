
\section{Measurement and Uncertainty\footnote{
1990-93 Dept. of Physics and Astronomy, Dickinson College. Supported by FIPSE
(U.S. Dept. of Ed.) and NSF. Portions of this material have been modified locally
and may not have been classroom tested at Dickinson College.
}}

\makelabheader %(Space for student name, etc., defined in master.tex or labmanual_formatting_commands.tex)

\textbf{Objectives} 

\begin{itemize}
\item To learn to measure length using a meter stick and a vernier caliper. 
\item To learn to express the results of measurements with the appropriate number
of significant figures. 
\item To learn how to compensate for systematic error in measurements so that accuracy
can be improved.
\end{itemize}
\textbf{Measuring Lengths and Significant Figures} 

We are interested in determining the number of significant figures in length
measurements you might make. How is the number of significant figures determined?
Suppose God could tell us that the ``true'' width of a certain
car key in centimeters was:

2.435789345646754456540123544332975774281245623... etc. 

If we were to measure the key width with a ruler that is lying around the lab,
the precision of our measurement would be limited by the fact that the ruler
only has lines marked on it every 0.1 cm. We could estimate to the nearest 1/100th
of a centimeter how far the key edge is from the last mark. Thus, we might agree
that the best estimate for the width of the key is 2.44 cm. This means we have
estimated the key width to three significant figures. 

If God announces that the width of a pair of sun glasses is 13.27655457787654267787...
cm, then upon direct measurement we might estimate the width to be 13.28 or
13.27 or 13.26 cm. In this case the estimated width is four significant figures.
Obviously, there is uncertainty about the ``true'' value of
the right-most digit.

The number of significant figures in a measurement is given by the number of
digits from the most certain digit on the left of the number up to and including
the first uncertain digit on the right. In reporting a number, all digits except
the significant digits should be dropped. (See the discussion of significant
figures in Appendix A.)

Let's do some length measurements to find out what factors might influence the
number of significant figures in a measurement. 

\textbf{Apparatus} 

\begin{itemize}
\item A meter stick 
\item A vernier caliper 
\item A rectangular board
\end{itemize}
\textbf{Activity 1: Length Measurements with the Meter Stick }

(a) What factors might make a determination of the ``true''
length of an object measured with the meter stick uncertain? 
\vspace{20mm}

(b) Measure the width of the board with the meter stick at least seven times
and create a table in the space below to list the measurements. For best results,
you should use different regions of the meter stick so that, when an average
of these measurements is made, non-uniformities in the scale will tend to cancel.
Also, you should avoid using the end of the meter stick which might be worn
and, therefore, would not be a true zero. The apparent change in the reading
on the scale due to the position of the eye is called parallax, an effect which
can introduce error into the reading. To reduce the uncertainty due to parallax,
you should place the meter stick on edge so that the scale is close to the object
being measured. (See Appendix E) 
\vspace{25mm}

(c) In general, when a series of measurements is made, the best estimate is
the average of those measurements. In the space below list the minimum measurement,
the maximum measurement, and the best estimate for the width of your board.
\vspace{20mm}

(d) Based on these measurements, write the width of your board as a value plus or minus an ``uncertainty''.
\vspace{15mm}

(e) How many significant figures should you report in your best estimate? Why?
\vspace{15mm}

(f) For your board, what limits the number of significant figures most - variation
in the actual width of the board or limitations in the accuracy of the meter
stick? How do you know?
\vspace{15mm}

\textbf{Activity 2: Length Measurements with the Vernier Caliper }

(a) Measure the thickness of the board at least seven times and record the results
in a table in the space below. Make the measurements at different places along
each of the two edges. If you have questions about how to read the vernier,
see Appendix E. If you still have questions, consult your instructor.
\vspace{25mm}

(b) Record the minimum, maximum, and average of your measurements below.
\vspace{20mm}

(c) Based on these measurements, write the thickness of your board as a value plus or minus an uncertainty.
\vspace{15mm}

(d) How many significant figures should you report in your average? Why? 
\vspace{15mm}

\textbf{Activity 3: Calculation of Cross-Sectional Area} 

(a) Calculate the cross-sectional area of the board in the space below.
\vspace{15mm}

(b) Calculate the uncertainty in the area, \( \Delta  A\), as follows:

   $$ \frac{\Delta A}{A} = \sqrt{\left(\frac{\Delta w}{w}\right)^2 + \left(\frac{\Delta t}{t}\right)^2}$$
\vspace{20mm}

(c) Write the cross-sectional area as a value plus or minus an uncertainty, rounding off as appropriate.
\vspace{15mm}

(d) How many significant figures should you report in your result? Why?
\vspace{15mm}

\textbf{The Inevitability of Uncertainty} 

In common terminology there are three kinds of ``errors'': (1)
mistakes or human errors, (2) systematic errors due to measurement or equipment
problems and (3) inherent uncertainties.

\textbf{Activity 4: Error Types} 

(a) Give an example of how a person could make a ``mistake''
or ``human error'' while taking a length measurement.
\vspace{15mm}

(b) Give an example of how a systematic error could occur because of the condition
of the meter stick when a set of length measurements are being made.
\vspace{20mm}

(c) What might cause inherent uncertainties in a length measurement?

