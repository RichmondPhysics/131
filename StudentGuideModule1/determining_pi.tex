
\section{Determining $\pi$}

Name \rule{2.0in}{0.1pt}\hfill{}Section \rule{1.0in}{0.1pt}\hfill{}Date \rule{1.0in}{0.1pt}

{\noindent \bf Objectives:}

\begin{list}{$\bullet$}{\itemsep0pt \parsep0pt}

\item Determine the relationship between circumference and diameter \item Understand the meaning of $\pi$

\end{list}

\textbf{Apparatus:}

\begin{itemize}
\item Wooden disks
\item Meter stick
\item String
\end{itemize}
{\noindent \bf Activity:}

\begin{enumerate}

\item With a meter stick, measure the diameter of one of the disks. Enter the value (or an average of several such measurements) in the table below.

\item With a string and meter stick, measure the circumference of the same disk, and enter the value (or average of several measurements) in the table below.

\item Repeat the previous steps 1. and 2. for each of the disks.

\begin{center} \begin{tabular}{|c|c|c|} \hline disk & diameter (cm) & circumference (cm) \\ \hline\hline 1 & & \\ \hline 2 & & \\ \hline 3 & & \\ \hline 4 & & \\ \hline 5 & & \\ \hline \end{tabular} \end{center}

\item Graph circumference versus diameter using your measurement data.

\item Fit data and determine the slope of the resulting line. \end{enumerate}

\vspace{10pt}

slope \rule{1.5in}{0.2pt}

\vspace{10pt}

{\noindent \bf Questions:}

1. What quantity is identified with the slope of the graph? 
\vspace{10mm}

2. Must the line go through the origin? Explain. 
\vspace{20mm}

3.If the diameter of the largest disk increased by a factor of 2.7, by how much
would its circumference change? 
\vspace{20mm}

4. If you formed a circle with a string 15 mm shorter than the circumference
of the smallest disk, its diameter would be how much smaller than the disk's?

