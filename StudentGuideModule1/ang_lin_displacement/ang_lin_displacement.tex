\section{Connecting Angular and Linear Displacement Along the Arc}

\makelabheader %(Space for student name, etc., defined in master.tex or labmanual_formatting_commands.tex)

\bigskip
\textbf{Apparatus}
\begin{itemize}[nosep]
\item Wooden disks
\item Meter stick
\item String
\end{itemize}

\bigskip
\textbf{Activity} 
\begin{enumerate}[labparts]
\item Use the string and meter stick to determine the circumference of the disk in cm, and calculate its diameter and radius.

\begin{center} 
circumference = \rule{2cm}{0.4pt} \hspace{1cm} diameter = \rule{2cm}{0.4pt} \hspace{1cm} radius = \rule{2cm}{0.4pt}
\end{center}

\item Check that the meter stick is inserted in the base with the scale visible.

\item Be sure the string is attached at the 0-mark on the disk. Wind the string around the disk one full rotation so that the 0-mark and the beginning of the string are at the bottom of the disk and the string is wound around once in such a way that when you unwind it, you can pull it along the meter stick and the disk will rotate counter-clockwise.

\item In this configuration, note the position of the tag on the string with reference to the meter stick.

\item While keeping the string taut, pull it gently to displace the tag five centimeters. Record in the table below the angular position of the bottom of the disk.

\item Repeat part (e) in five centimeter increments until the string is completely unwound.

\begin{center}
\renewcommand{\arraystretch}{1.8}{
\begin{tabular}{|C{0.8in}|C{1.0in}|C{1.0in}|} \hline 
$\Delta s$ (cm) & $\theta$ (rad) & $\Delta\theta$ (rad)\\ 
\hhline{|=|=|=|}
5.0&& \\ \hline 
10.0&& \\ \hline 
15.0&& \\ \hline 
20.0&& \\ \hline 
25.0&& \\ \hline 
30.0&& \\ \hline 
35.0&& \\ \hline 
40.0&& \\ \hline 
45.0&& \\ \hline 
\end{tabular}}
\end{center}

\item Calculate the angular displacements and plot linear displacement 
($\Delta s$) on the $y$ axis versus angular displacement ($\Delta\theta$) on the 
$x$ axis.

\end{enumerate}

\pagebreak[2]

\textbf{Questions:}

\begin{enumerate}[labparts]
\item Do your data points appear to fall on a line? Did you expect them to? Explain. \vspace{15mm}

\item If you were to fit a line through these points, would that line go through the
origin? Explain. \vspace{15mm}

\item Fit a straight line through the data. Express in words the meaning of 
the slope. What is its physical interpretation? \vspace{15mm}

\item If a bigger disk were used, what would your result be? For a given angular displacement
of the bigger disk, how would the linear displacement differ? \vspace{15mm}

\item What is the ratio of some arc length to the radius of the disk? \vspace{15mm}

\item The unit given for the angular displacement is rad, which is an abbreviation
for radians. Express, in words, the meaning of the term. \vspace{15mm}

\item How many radians are in half the circumference of the disk? In the whole circumference?
\end{enumerate}
