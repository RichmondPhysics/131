\section{Vernier Calipers}
\label{vernier_caliper}

A vernier is a small auxiliary scale that slides along the main scale. It allows more accurate estimates of fractional parts of the smallest division on the main scale.

On a vernier caliper, the main scale, divided into centimeters and millimeters, is engraved on the fixed part of the instrument. The vernier scale, engraved on the movable jaw, has ten divisions that cover the same spatial interval as nine divisions on the main scale:\hspace{4pt} each vernier division is $\frac{9}{10}$ the length of a main scale division. In the case of a vernier caliper, the vernier division length is 0.9 mm. [See figures below.]

\begin{center} \setlength{\unitlength}{1pt} \begin{picture}(330,125) \put(2,120){\makebox(0,0)[bl]{0}} \put(102,120){\makebox(0,0)[bl]{1}} \multiput(10,100)(10,0){12}{\line(0,1){10}} \multiput(9,96)(9,0){9}{\line(0,1){4}} \put(50,100){\line(0,1){15}} \put(45,93){\line(0,1){7}} \put(50,75){\makebox(0,0)[b]{0.00}}

\put(202,120){\makebox(0,0)[bl]{0}} \put(302,120){\makebox(0,0)[bl]{1}} \multiput(210,100)(10,0){12}{\line(0,1){10}} \multiput(214,96)(9,0){9}{\line(0,1){4}} \put(250,100){\line(0,1){15}} \put(250,93){\line(0,1){7}} \put(250,75){\makebox(0,0)[b]{0.05}}

\put(84,45){\makebox(0,0)[bl]{1}} \put(184,45){\makebox(0,0)[bl]{2}} \multiput(92,25)(10,0){13}{\line(0,1){10}} \multiput(114,21)(9,0){9}{\line(0,1){4}} \put(132,25){\line(0,1){15}} \put(150,18){\line(0,1){7}} \put(150,0){\makebox(0,0)[b]{1.23}}

\put(150,-20){\makebox(0,0)[b]{Examples of vernier caliper readings}}

\thicklines \multiput(0,100)(100,0){2}{\line(0,1){20}} \multiput(0,90)(90,0){2}{\line(0,1){10}} \put(0,100){\line(1,0){130}}

\multiput(200,100)(100,0){2}{\line(0,1){20}} \multiput(205,90)(90,0){2}{\line(0,1){10}} \put(200,100){\line(1,0){130}}

\multiput(82,25)(100,0){2}{\line(0,1){20}} \multiput(105,15)(90,0){2}{\line(0,1){10}} \put(80,25){\line(1,0){150}}

\end{picture} \end{center} \medskip

To measure length with a vernier caliper, close the jaws on the object and read the main scale at the position indicated by the zero-line of the vernier. The fractional part of a main-scale division is obtained from the first vernier division to coincide with a main scale line. [See examples above.]

If the zero-lines of the main and vernier scales do not coincide when the jaws are closed, all measurements will be systematically shifted. The magnitude of this shift, called the zero reading or zero correction, should be noted and recorded, so that length measurements made with the vernier caliper can be corrected, thereby removing the systematic error.

