
\section{Kepler's Laws\footnote{
Adapted from \underbar{Activities in Astronomy}, D. Hoff, L. Kelsey, and J.
Neff, Kendall/Hunt Publishing Co., 1978.
}}

Name \rule{2.0in}{0.1pt}\hfill{}Section \rule{1.0in}{0.1pt}\hfill{}Date \rule{1.0in}{0.1pt}

{\noindent \bf Objectives:} \begin{list}{$\bullet$}{\itemsep0pt \parsep0pt}

\item The applicability of Kepler's Laws \item The mass of the moon

\end{list}

\noindent {\bf Introduction} 

\noindent Based on data provided by Tycho Brahe, Johannes Kepler devised three laws of planetary motion:

1. The orbits of the planets are ellipses with the sun at one focus.

2. A line segment connecting the sun and a given planet sweeps out equal areas in equal time intervals.

3. The square of the period is proportional to the cube of the semi-major axis of the orbit.

\noindent Newton's Law of Gravitation generalized these laws to apply to any two bodies in orbital motion about one another.

\noindent Newton's analysis led to the following formulation of Kepler's Third Law:

\begin{equation} T^2 = \frac{4\pi^2}{G(M + m)}a^3 \end{equation}

\noindent where $M$ and $m$ are the two masses, $T$ is the period of mutual revolution, $G$ is the universal gravitational constant, and $a$ is the semi-major axis of their relative orbit. \\

{\noindent \bf Apparatus:} \begin{list}{$\bullet$}{\itemsep0pt \parsep0pt}

\item ruler

\end{list}

{\noindent \bf Activity 1: The ellipse} \\

\noindent {\bf Discussion} An ellipse is the set of points in the plane for which the sum of the distances from two fixed points is constant. The two points are called foci (the plural of focus). If the center of the ellipse is at the origin, then the standard form of the equation of an ellipse is

\begin{equation} \frac{x^2}{a^2} + \frac{y^2}{b^2} = 1 \end{equation}

\noindent where the curve intercepts the $x$-axis at $(a,0)$ and $(-a,0)$, and the $y$-axis at $(0,b)$ and $(0,-b)$. If $a > b$, then the foci are at $(\pm c,0)$, where $c = \sqrt{a^2 - b^2}$. If $a < b$, then the foci are at $(0,\pm c)$, where now $c = \sqrt{b^2 - a^2}$. If $a = b$, the ellipse is a circle. The departure from circularity is quantified with the term eccentricity, defined symbolically as

\begin{equation} \varepsilon = \frac{c}{a} \end{equation}

\noindent when $a > b$. If $P$ is any point on the ellipse, then the sum of its distances from the foci is $2a$ for $a > b$, or $2b$ for $a < b$. An ellipse can be drawn with a string tacked at the foci, pulling the string taut with a pencil, and running the pencil around. \\

\begin{enumerate}

\item Sketch an ellipse with an eccentricity of 1.

\vskip35pt

\item Sketch an ellipse with an eccentricity of 0.

\vskip35pt

\end{enumerate}

{\noindent \bf Activity 2: Explorer 35 and Kepler's Laws}\\

\noindent {\bf Discussion} Explorer 35 lifted off from Cape Kennedy on 19 July 1967, and entered orbit around the moon on 22 July. The satellite, weighing 230 lbs. (1.04.4 kg), carried instruments for measuring solar x-rays and energetic particles; the solar wind in interplanetary space and its interaction with the moon; and the magnetic properties of the moon. It continued operation until June 1973, having accomplished all its mission objectives.

\noindent The data\footnote{From the NASA-Goddard Space Flight Center.} below form a sample set of positions of Explorer 35 in its elliptical orbit about the moon. The time interval between position measurements is 15 minutes. The unit of length is the radius of the moon. The center of the coordinate system is the center of the moon. \\

\begin{enumerate}

\item Using Excel, plot the data and find the major axis and foci.

\item Find the semi-major axis, the minor axis, the semi-minor axis, the eccentricity, and the period of the orbit (this last one, by inspection).

\vskip35pt

\item Verify Kepler's First Law. [Hint: refer to the definition of an ellipse.]

\vskip35pt

\item Determine the areas swept out in a time interval at different points of the orbit and show that Kepler's Second Law is valid. [Hint: use triangles as an approximation.]

\vskip35pt

\item Using the Newtonian form of Kepler's Third Law, find the mass of the moon in kilograms. [radius of the moon = 1738 km; and $G = 6.668 \times 10^{-8}$ cm$^3$/gm sec$^2$ $= 8.642 \times 10^{-13}$ km$^3$/kg hr$^2$]

\end{enumerate}

\vfill \pagebreak 

\textbf{Questions:} 

\begin{enumerate}
\item Discuss variations in your measurements used to validate the First and Second
Laws.\vspace{30mm}

\item Noting the number of significant figures in the data, how do the variations
in your measurements compare to the uncertainties in the data?\vspace{30mm}

\item Estimate the uncertainty in your determination of the mass of the moon. What
has a greater effect, uncertainties in \( a \) or those in \( T \)?
\end{enumerate}
\vfill \pagebreak

\begin{center} \begin{tabular}{c c c c|c c c c} \hline \multicolumn{2}{c}{Elapsed} & X & Y & \multicolumn{2}{c}{Elapsed} & X & Y \\ \multicolumn{2}{c}{time} & & & \multicolumn{2}{c}{time} & & \\ (h) & (m) & (lunar radii) & (lunar radii) & (h) & (m) & (lunar radii) & (lunar radii) \\ \hline 0 & 00 & -3.62 & 1.04 & 6 & 00 & 0.27 & 4.86 \\ 0 & 15 & -3.46 & 0.63 & 6 & 15 & -0.56 & 4.95 \\ 0 & 30 & -3.25 & 0.20 & 6 & 30 & -0.84 & 5.01 \\ 0 & 45 & -2.97 & -0.22 & 6 & 45 & -1.12 & 5.03 \\ 1 & 00 & -2.60 & -0.65 & 7 & 00 & -1.38 & 5.04 \\ 1 & 15 & -2.14 & -1.03 & 7 & 15 & -1.64 & 5.00 \\ 1 & 30 & -1.55 & -1.37 & 7 & 30 & -1.89 & 4.95 \\ 1 & 45 & -0.85 & -1.58 & 7 & 45 & -2.14 & 4.87 \\ 2 & 00 & -0.03 & -1.59 & 8 & 00 & -2.37 & 4.77 \\ 2 & 15 & +0.78 & -1.32 & 8 & 15 & -2.59 & 4.65 \\ 2 & 30 & 1.45 & -0.79 & 8 & 30 & -2.80 & 4.50 \\ 2 & 45 & 1.87 & -0.11 & 8 & 45 & -2.99 & 4.33 \\ 3 & 00 & 2.09 & +0.58 & 9 & 00 & -3.17 & 4.14 \\ 3 & 15 & 2.16 & 1.22 & 9 & 15 & -3.33 & 3.93 \\ 3 & 30 & 2.11 & 1.82 & 9 & 30 & -3.49 & 3.69 \\ 3 & 45 & 1.99 & 2.35 & 9 & 45 & -3.59 & 3.42 \\ 4 & 00 & 1.82 & 2.81 & 10 & 00 & -3.69 & 3.15 \\ 4 & 15 & 1.61 & 3.22 & 10 & 15 & -3.77 & 2.85 \\ 4 & 30 & 1.37 & 3.59 & 10 & 30 & -3.81 & 2.52 \\ 4 & 45 & 1.11 & 3.90 & 10 & 45 & -3.83 & 2.20 \\ 5 & 00 & 0.85 & 4.16 & 11 & 00 & -3.81 & 1.83 \\ 5 & 15 & 0.58 & 4.40 & 11 & 15 & -3.76 & 1.46 \\ 5 & 30 & 0.28 & 4.58 & 11 & 30 & -3.65 & 1.06 \\ 5 & 45 & 0.00 & 4.74 & 11 & 45 & -3.51 & 0.65 \\ \hline

\end{tabular} \end{center}

