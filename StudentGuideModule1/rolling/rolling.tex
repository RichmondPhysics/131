
\section{Rolling}

\makelabheader %(Space for student name, etc., defined in master.tex or labmanual_formatting_commands.tex)

\textbf{Objectives }

To apply our understanding of rotation to the case of rolling without skidding
and discover its fundamental properties. 

\textbf{Apparatus}

\begin{itemize}
\item A spool for rolling. 
\item A meter stick and a ruler 
\item A video analysis system (\textit{VideoPoint}).
\end{itemize}
\textbf{Overview }

We have extensively studied the linear motion of objects and we recently began
the study of rotational motion. When something rolls it is combining both types
of motion in one. In addition, the motion requires a unique application of forces
and it has some surprising characteristics that are often poorly understood.
In this unit we will investigate these unique features.

\textbf{Activity 1: Rolling --- Guesses and Predictions }

To begin our investigation consider the following questions. You might find
it useful to test some of your answers by simply rolling the spool along the
table and observing its motion.

\begin{enumerate}
\item In your own words, what is rolling?
\answerspace{20mm}

\item How is the rotational motion related to the linear motion?
\answerspace{20mm}

\item Make a sketch of the spool rolling and the forces acting on the spool. What
is the effect, if any, of friction? How would the motion change if there was
no friction in the system at all?
\answerspace{30mm}
\end{enumerate}

\pagebreak[2]
\textbf{Activity 2: Investigating Rolling} 

(a) Make a movie of the spool rolling along your laboratory table by following
these steps. 

\begin{enumerate}
\item Turn the camera on and point it along the long axis of your table. The camera
should be around 1 m from where you will roll the spool along the surface of
the table. Place a ruler somewhere in the field of view where it won't interfere
with the motion and parallel to one edge of the field of view. This ruler will
be user later to determine the scale. 
\item Practice rolling the spool in front of the camera a few times to make sure the
camera is pointing so there is no change in the vertical position of the center
of the spool as it moves across the camera's field of view. Make a movie of
the spool rolling at a constant speed along the table. See \textbf{Appendix
\ref{videopoint}: Video Analysis} for details on making the movie.
\end{enumerate}
(b) Determine the position of a point on the rim of the spool and the center
of the spool during the motion. To do this task follow the instructions in \textbf{Appendix
\ref{videopoint}: Video Analysis} for creating and calibrating movie data. On each frame click
first on the point on the rim and then on the center. Make sure the vertical
position of the center did not change much during the movie. If it did, check
the orientation of your camera and retake the movie. The file should contain
five columns with the values of time, x and y positions of the point on the
rim, and x and y positions of the center of the spool.

(c) Create a plot of the horizontal position of the center of the spool versus
time. See \textbf{Appendix \ref{excel}: Introduction to Excel} for more details.
Fit your data and extract the speed of the center of the spool. Print your plot
and attach it to your write-up. Measure the radius of the spool.
\vspace{5mm}

\( v_{\mbox{\small spool}} \) = \hfill{}\( r_{\mbox{\small spool}} \) = \hfill{}
\vspace{5mm}

(d) What is the general expression for the angular displacement of the point
on the rim relative to the center of the spool in terms of the x and y positions
that you recorded above? 
\vspace{5mm}

\( \theta _{\mbox{\small rim}}= \) 
\vspace{5mm}

(e) We want to graph the angular displacement of the point on the rim relative
to the center as a function of time. To do this:

\begin{enumerate}
\item Calculate the angular displacement \( \theta _{\mbox{\small rim}} \) in radians for the first
row in your data table. Record the result here.

$\theta _{1}=$

\item Calculating the angular displacement for all the data as we just did
would be horribly tedious. Instead, use an {\it Excel} formula to
figure out the angular displacements.  (See Appendix \ref{excel} for details.)
You may find it helpful to know that there is an {\it Excel} function
ATAN2 that takes the inverse tangent of the ratio of two numbers.
For instance, if you put ``=ATAN2(C3,D3)'' into a cell, {\it Excel}
will calculate the inverse tangent of the ratio of the number
in cell D3 to the number in cell C3.  (Note that the ratio
that is taken has the second argument on top: in this case, it's 
D3/C3, not C3/D3.)


\item Graph the angular displacement as a function of time.
You will see discontinuous jumps in your data because the function you used
in part (3) always calculates angles in the range \( -\pi  \) to \( \pi  \).
You must add different increments of \( 2\pi  \), \( 4\pi  \), etc. to adjust
the scale of the angular displacement. Do this, print your plot and attach it
to your write-up. 
\end{enumerate}

(f) Fit your graph of the angular displacement versus time and extract the angular
speed. Calculate the ratio of the speed of the center of the spool and the angular
speed. What are the units of this ratio? Is it close in value to any property
of the spool? Is it related to any property of the spool? Why?
\vspace{5mm}

\( \omega _{\mbox{\small rim}}= \)\hfill{} \( v_{\mbox{\small spool}}/
\omega _{\mbox{\small rim}}= \)\hfill{}
\vspace{10mm}

(g) Make a plot of the vertical and horizontal components of the position of
the point on the rim as a function of time. Attach a copy of your plot to the
unit. What is the speed of the point on the rim when it is in contact with the
table? Explain.
\vspace{20mm}

(h) How would you define rolling now? What is the role of friction?

