\section{Drag Force and Terminal Velocity}

\begin{comment}
By Matt Trawick, 11/2019 (though surely many people have done similar things before this!)

\end{comment}

\makelabheader %(Space for student name, etc., defined in master.tex or labmanual_formatting_commands.tex)

\medskip
\textbf{Apparatus}

\begin{itemize}[nosep]
\item high precision digital lab scale
\item ruler
\item about 5 coffee filters
\item motion sensor, mounted high on lab stand, pointed down
\item \textit{Capstone} software (\filename{P\_V\_A\_graphs.cap} experiment file)
\end{itemize}

\medskip
\textbf{Activity 1: Predictions and First Data}

\begin{enumerate}[labparts]
\item A ``drag force'' $F_{\rm drag}$ is a force caused by the air hitting against a moving object.  You would feel a drag force if you held your arm out of a moving car.\footnote{Don't try this at home.  Keep arms and legs inside moving vehicle at all times.}  If the car's speed increased, would the drag force on your hand increase, decrease, or stay the same?
\answerspace{0.3in}

\item Suppose you drop an object, like a ball or a tissue.  Draw a free body diagram for the object as it falls, including the drag force.
\answerspace{1.2in}

\item Are the two forces in your diagram always constant during the fall?  Or will either of them change over time?  (Hint: does the object fall with constant speed?)
\answerspace{0.5in}

\item Would the falling object continue to accelerate if the drag force and the gravitational force were equal?
\answerspace{0.5in}

\item On the axes below, draw a sketch with a dashed line predicting the shape of the velocity vs. time graph for a falling object (such as a ball or a tissue), including the effect of a drag force.  Assume the falling object starts from rest, $v=0$.

\begin{lab_axis}*[lab_noticks_1quad,
	height = {1.2in}, width = {3.0in},
	xlabel={Time},
	ylabel={Velocity},
	]
\end{lab_axis}

\item Now let's do the experiment.  Open the file \filename{P\_V\_A\_graphs.cap}, located in the \filename{\coursefolder} folder. Use a stack of three coffee filters nested inside each other, dropping them from just below the motion detector to obtain a graph of $v(t)$.  You may need to try this a few times to get a smooth graph.  Set the switch on the motion detector to the wide beam setting.  If your stack of filters flutters side to side, try bending the edges up to make the filters less flat.  Once you have a good velocity graph, sketch its shape with a solid line on the axes below.  

\begin{lab_axis}*[lab_grid,
	xlabel={Time (s)},
	ylabel={Velocity (m/s)},
	width=4.5in, height=1.8in,
	xmin=0, xmax=5,
	xtick distance = 1,
%	ytick distance = 1,
	minor x tick num=1,
%	minor y tick num=1,
	]
\end{lab_axis}

\item Indicate on your graph the place where $F_{\rm drag} = F_{\rm grav}$.  

\item The maximum speed of the falling object, reached when $F_{\rm drag} = F_{\rm grav}$ is called the ``terminal velocity'' $v_T$.  What is $v_T$ for your stack of three coffee filters?
\answerspace{0.5in}

\end{enumerate}

\medskip
\textbf{Activity 2: Drag Force vs. Velocity}

%You found in the previous activity that the maximum speed, or ``terminal velocity'' $v_T$ 
%is reached when  $F_{\rm drag} = F_{\rm grav}$.  
In this activity, you will measure 
$F_{\rm drag}$ as a function of velocity $v$ by recording the terminal velocity $v_T$ for stacks of filters with different weights.

\begin{enumerate}[labparts]
\item Measure the terminal velocity $v_T$ for a single coffee filter, and also for nested stacks of up to five coffee filters.  
Try to make sure the shape of each falling stack is always the same; bend the edges up or down as needed.
Record your results in the table below.

\begin{center}
{\renewcommand{\arraystretch}{1.8}
\begin{tabular}{|c | c| C{1in} |}
\hline
$n$ & $F_{\rm drag}~ (= F_{\rm grav}$) at $v_T$ & $v_T$ \\ 
\hhline{|=|=|=|}
1 & & \\ \hline
2 & & \\ \hline
3 & & \\ \hline
4 & & \\ \hline
5 & & \\ \hline
\end{tabular}
}
\end{center}

\pagebreak[2]
\item Copy your data from the table above into Excel.  Make a log-log graph of $F_{\rm drag}$ vs. $v_T$, plotting the force on the $y$~axis and the speed on the $x$~axis.
Note the twist here: your goal is to understand how the drag force $F_{\rm drag}(v)$ varies as a function of speed $v$.
This is conceptually the reverse of the experiment you did, where the force was more like the \textit{independent} variable (which you set by changing $F_{\rm grav}$, by altering the number of filters) and the terminal velocity was more like the \textit{dependent} variable which you measured as a result.

\item For very low speeds, or for objects moving in incompressible fluids like water, the drag force is typically proportional to $v$.  In many other cases, the drag force is often proportional to $v^2$.  Does your data suggest a power law relationship $F_{\rm drag} \propto v^p$?  Use a fit to determine what exponent $p$ is most consistent with your data.  Write your value of $p$ below, and also print a copy of your graph if your instructor requests it.
\answerspace{0.8in}

For objects in air at moderate speeds, the drag force on an object can be expressed as
\begin{equation}
F_{\rm drag} = \frac{1}{2}C \rho A v^2,
\end{equation}
where $\rho$ is the density of air (1.2~kg/m$^3$ at room temperature and atmospheric pressure), 
$A$ is the object's cross-sectional area, and 
$C$ is a dimensionless ``drag coefficient'' based on the object's shape.  
You can think of $C$ as a measure of how ``aerodynamic'' an object is.  The table below shows a few examples.

\begin{center}
{\renewcommand{\arraystretch}{1.2}
\begin{tabular}{|c | c |}
\hline
object & drag coefficient $C$ \\ 
\hhline{|=|=|}
sphere & 0.47 \\ \hline
cube & 1.05 \\ \hline
Toyota Prius & 0.25 \\ \hline
typical pickup truck & 0.40 \\ \hline
\end{tabular}
}
\end{center}

\item What is the cross-sectional area $A$ of your stack of coffee filters?  Note that this is NOT the actual surface area of the filters.  It's more like the area of the \textit{shadow} cast by the filters if light were shining in the direction of motion.
\answerspace{0.8in}

\item Perform a fit to your data based on Equation (1) above.  Based on your data, what is the drag coefficient $C$ for your stack of coffee filters?  Also check that your number seems sensible in light of the other values of $C$ in the table above.  Is your stack of coffee filters as ``aerodynamic'' as a Toyota Prius?
\answerspace{0.8in}


\end{enumerate}

