\usepackage{pgfplots}
\usepackage{sansmath}
\usepgfplotslibrary{groupplots}
\usetikzlibrary{pgfplots.groupplots}
\usetikzlibrary{math}
\usepackage{xparse} %used for \NewDocumentCommand
\listfiles

\pgfplotsset{compat=1.16}
\tikzset{every picture/.style={font=\sffamily}}

\pgfplotscreateplotcyclelist{lab_solid}{
	{black}
}

\pgfplotsset{
  tick label style = {font=\sansmath\sffamily},
%  every axis label = {font=\sansmath\sffamily},
%  legend style = {font=\sffamily},
%  label style = {font=\sffamily}
	samples = 200,
}


\pgfplotsset{lab_noticks_1quad/.style={
	cycle list name = lab_solid,
	every axis plot/.append style={very thick},
	axis x line={left, x axis line style={-}}, 
	axis y line= {left,  y axis line style={-}},
	xmin=0, xmax=1,
	ymin=0, ymax=1, %Although there are no ticks, this seems to be needed to make the ticks disappear.
	tick style={xtick=\empty,ytick=\empty}
}}

\pgfplotsset{lab_noticks_2quads/.style={
	cycle list name = lab_solid,
	every axis plot/.append style={very thick},
	axis x line= {middle, x axis line style={-}}, 
	axis y line= {left,  y axis line style={-}},
      xlabel style={at=(current axis.right of origin), anchor=west},
	xmin=0, xmax=1,
	ymin=-1, ymax=1, 
	tick style={xtick=\empty,ytick=\empty},
}}

\pgfplotsset{lab_noticks_4quads/.style={
	cycle list name = lab_solid,
	every axis plot/.append style={very thick},
	axis x line= {middle, x axis line style={-}}, 
	axis y line= {middle,  y axis line style={-}},
      xlabel style={at=(current axis.right of origin), anchor=north east},
      ylabel style={at=(ticklabel* cs:1.0), anchor=north east},
	xmin=-1, xmax=1,
	ymin=-1, ymax=1, 
	tick style={xtick=\empty,ytick=\empty},
}}

\pgfplotsset{plus_minus_zero_labels/.style={
	tick style={major tick length=0pt},
	extra y ticks = {-0.8, 0, 0.8},
	extra y tick labels = {\scriptsize{$-$},0,\scriptsize{$+$}},
}}

\pgfplotsset{lab_grid/.style={
	cycle list name = lab_solid,
	every axis plot/.append style={very thick},
	tick style={grid=major,major tick length=0pt}
}}


% The commmand below makes vertical alignment marks through all graphs in a groupplot.
% (Restricted to only a single column, but multiple rows.  The rows can be 1 though.)
%
% How to call:
% \makegroupverticlas[rows]{x1,x2,x3...}{xmin}{xmax}
% rows is assumed to be 2 if not specified.  xmin and xmax are the min and max values of the axis.
% Oh, and it won't work if you define a name for the group; it must remain the default "group".
%
% This was a giant pain to get to work.  
% The problem seems to be that INSIDE the groupplot environment, tikz knows about xmin and xmax
% (since it can get it from \pgfkeysvalueof{/pgfplots/xmin) but does not know about the top and bottom
% of the whole group; only of a single axis.
% But OUTSIDE the grouplot environment, tikz knows about the top and bottom of the group
% (from group c1r1.north west, for instance), but does not know about the scale of the x axis.
% Or at least I couldn't figure out how to get it to know about the x axis.  Maybe someone else can.
% If so, it would be a nice improvement to NOT have to pass it xmin and xmax, or even the number of
% rows, for that matter. 
\NewDocumentCommand \makegroupverticals {o m m m} {
	\coordinate (TL) at (group c1r1.north west); % top left
      \IfNoValueTF  {#2} {%If no lastrow given, assume two rows
		\coordinate (BR) at (group c1r3.south east); % bottom right
		}
	{ % else use the value for the number of the last row
		\coordinate (BR) at (group c1r#1.south east); % bottom right
		}
	\newdimen\xleft
	\pgfextractx\xleft{\pgfpointanchor{TL}{center}}
	\newdimen\xright
	\pgfextractx\xright{\pgfpointanchor{BR}{center}}

	\newdimen\xx
	\foreach \x in {#2} { 
		\tikzmath{\xx = \xleft + (\xright - \xleft) * ((\x - #3) / (#4 - #3))  ;}
		\draw[black, thick, dashed] (\xx, |- TL) -- (\xx, |- BR);
		}
}

% Here's an older version of makegroupverticals.
% It does not work, because the x axis coordinates in \x are often wrong,
% Sometimes by a huge amount.  I suspect there's a way to make this work,
% with a small fix, so I'll keep it here.
\NewDocumentCommand \oldmakegroupverticals {o m} {
	\coordinate (toppoint) at  (group c1r1.north);
      \IfNoValueTF  {#2} {%If no lastrow given, assume two rows
		\coordinate (botpoint) at  (group c1r2.south);
		}
	{ % else use the value for the number of the last row
		\coordinate (botpoint) at  (group c1r#1.south);
	}
	\foreach \x in {#2} 
		\draw [black, very thick, dashed] (\x, |- toppoint) -- (\x, |- botpoint);
}

\NewDocumentEnvironment{lab_axis}{s}{
	\IfBooleanTF #1
		{ \begin{center} } %if starred
		{} %if not starred, do nothing
	\vspace{1em}
	\begin{tikzpicture} 
	\begin{axis}
	}
	{\end{axis} 
	\end{tikzpicture}
	\vspace{1em}
	\IfBooleanTF #1
		{\end{center} } %if starred
		{} %if not starred, do nothing
	 }
% Next four lines allow either 
%	\begin{labaxis*}... \end{labaxis*} 
% or 
% 	\begin{labaxis}*... \end{labaxis}
\ExplSyntaxOn
\cs_new:cpn {lab_axis*} {\lab_axis*}
\cs_new_eq:cN {endlab_axis*} \endlab_axis
\ExplSyntaxOff

%This next part doesn't seem to be working yet.
\NewDocumentEnvironment{lab_groupplot}{s m}{
	\IfBooleanTF #1
		{ \begin{center} } %if starred
		{} %if not starred, do nothing
	\begin{tikzpicture}
	\begin{groupplot}
	}
	{\end{groupplot}
	#2
	\end{tikzpicture}
	\IfBooleanTF #1
		{\end{center} } %if starred
		{} %if not starred, do nothing
	 }
% Next four lines allow either 
%	\begin{labaxis*}... \end{labaxis*} 
% or 
% 	\begin{labaxis}*... \end{labaxis}
\ExplSyntaxOn
\cs_new:cpn {lab_groupplot*} {lab_groupplot*}
\cs_new_eq:cN {endlab_groupplot*} \endlab_groupplot
\ExplSyntaxOff

