\section{Hooke's Law}

\instructornote{%
Hooke's law is covered elsewhere in the 131 manual, in Activites 1 \& 2 of ``Work and Kinetic Energy,'' 
and again (briefly) in Activity 3 of ``Periodic Motion.''  It may not be necessary to include it in the manual as
a standalone lab.
}

\makelabheader %(Space for student name, etc., defined in master.tex or labmanual_formatting_commands.tex)

\bigskip
\textbf{Apparatus}
\begin{itemize}[nosep]
\item Two springs and supports 
\item Collection of masses 
\item Two-meter stick
\end{itemize}

\bigskip
\textbf{Introduction}

There is no such thing as a perfectly rigid body. The stiffest of metal bars can be twisted, bent, stretched, and compressed. Delicate measurements show that even small forces cause these distortions. Under certain circumstances (typically, when the forces are not too large), a body deformed by forces acting upon it will return to its original size and shape when the forces are removed, a capacity known as elasticity. Permanent distortion from large forces is referred to as plastic deformation. In this lab, you will stay within the elastic limit.


\bigskip
\textbf{Activity} 
\begin{enumerate}[labparts]

%\item  Record the mass of the two springs.

%\begin{displaymath} m_1 = \hskip50pt  m_2 = \end{displaymath}

\item  Suspend one of the springs from the support. Using the meter stick, observe the position of the lower end of the spring and record the value in the table below.

\item  Hang 50 grams from the lower end of the spring and again record the position of this end.

\item  Repeat part (b) with loads of 100, 200, 300, and 400 grams hung from the spring.

\item  Repeat parts (b) and (c) with the second spring.

\begin{center} 
{\renewcommand{\arraystretch}{1.6}
\begin{tabular}{|C{0.5in}|C{0.7in}|C{0.7in}|C{0.7in}  ||  C{0.5in}|C{0.7in}|C{0.7in}|} 
\hline
Mass on spring 1 & Force on spring 1 (N) & Position reading (m) & Elongation (m) &
Mass on spring 2 & Position reading (m) & Elongation (m) \\
\hhline{|=|=|=|=#=|=|=|}
0 & 0 & & 0 & 0 & & 0 \\ \hline 
50 & & & & 50 & & \\ \hline 
100 & & & & 100 & & \\ \hline 
200 & & & & 200 & & \\ \hline 
300 & & & & 300 & & \\ \hline 
400 & & & & 400 & & \\ \hline 
\end{tabular} }
\end{center}

\item Determine the elongation produced by each load.

\item  Plot a curve using the values of the elongation as the abscissas ($x$ values) and the forces due to the corresponding loads as ordinates ($y$ values) for each spring. Make sure you use compatible units. Write the equation for each curve in the space below.

\answerspace{0.8in}

\end{enumerate}

\pagebreak

\textbf{Questions}
\begin{enumerate}[labparts]
\item What do your curves show about the dependence of each spring's elongation upon the applied force?
\vspace{30mm}

\item List the proportionality constant (including proper units) for each spring in the space below.
\vspace{30mm}

\item The slope of each line (the proportionality constant) is known as the 
force constant, $k$. Use the LINEST function in \textit{Excel} (see Appendix \ref{excel}:
 Excel) to determine the slope of the line and the uncertainty in the slope. 
Write the force constant as $k$ = $slope$ \( \pm \ \Delta  slope\).  Be sure to 
include the proper units.
\end{enumerate}
