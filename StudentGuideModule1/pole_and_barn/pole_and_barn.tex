\section{The Pole and Barn Paradox}

\makelabheader %(Space for student name, etc., defined in master.tex)

\instructornote{%
By Matt Trawick, added to 131 in 2019.  Time: 35 minutes

This lab has been used in Modern for many years before 2019.  It assumes students are familiar with the ideas of length contraction and Lorentz transforms.  It doesn't give a lot of introduction to the Mathematica simulation, so it probably kind of assumes they've seen that before in previous labs too.  Good preparation for this would be the lab ``Galilean Transformations and Lorentz Transformations.''

Otherwise, there is no specific introduction necessary for this lab; it is self-contained, in the sense that you don't need to explain this specific paradox before they start.
}

\bigskip
\textbf{Apparatus:}
\begin{itemize}[nosep]
\item \textit{Mathematica}, with notebook file \filename{pole\_and\_barn.nb}
\end{itemize}

\bigskip

An unrealistically speedy pole vaulter named Anna runs with her 10 meter long pole at $0.8 c$ in the positive $x$ direction. She runs into an 8~meter-long barn, which has doors on both sides.  The right end of the pole enters the barn at $x = 0$, at some time $t < 0$, and the left end of the pole enters the barn at $x = 0$ and $t = 0$.

Here's what happens, according to farmer Bob, standing inside the barn:
\begin{itemize}[nosep]
\item The pole is length-contracted so that it's shorter than the barn.
\item When the pole is centered inside the barn, Bob pushes a button so that both doors shut very quickly with the pole entirely inside the barn.
\item Bob says, ``Ah-ha! I've closed Anna's pole inside the barn.''
\end{itemize}

But here's what happens according to Anna:
\begin{itemize}[nosep]
\item Her pole is 10 meters long.
\item The barn is length-contracted to less than 8 meters long.
\item Anna says, ``My pole couldn't possibly have been closed inside the barn!''
\end{itemize}

This seems like a paradox, since both observers have to agree on any tangible outcome of an experiment!

\begin{enumerate}[wide, nosep]
\item Open the file \filename{pole\_and\_barn.nb} in Mathematica. If you get a pop-up error message, you may need to click \button{Enable Dynamic Content}. Type \button{Ctrl-A} to select all lines, and hit \button{Shift-Enter} to execute them. The graph you see represents a ``spacetime diagram'' (or ``Minkowski diagram'') of this set of events, with $x$ on the horizontal axis and $t$ on the vertical axis. If the units on the horizontal axis are in meters, and the diagonal dotted lines represent objects traveling from the origin at the speed of light, what are the units of the vertical (time) axis?
\answerspace{0.6in}

\item With the velocity slider set to $v = 0$ (the default), the diagram is in the reference frame of farmer Bob. The red lines represent the positions in time (so-called ``worldlines'') of the two barn doors, and the purple lines represent the positions in time of the two ends of Anna's pole. Based on this diagram, what is the approximate length of Anna's pole according to Bob?  \label{part_approx_pole_length}
\answerspace{0.6in}

\item Verify your answer in part~\ref{part_approx_pole_length} by doing a calculation to determine the precise length contraction of the pole according to Bob.
\answerspace{0.6in}

\item Now, move the slider so that you view these events from Anna's reference frame.
From the graph, how long is the barn according to Anna? \label{part_approx_barn_length}
\answerspace{0.6in}

\item Verify your reading of the graph in part~\ref{part_approx_barn_length} by calculating the precise length of the barn in Anna's reference frame.
\answerspace{0.6in}

\item Now move the slider back to Bob's reference frame. According to Bob, at what time is the pole exactly centered inside the barn?
\answerspace{0.6in}

\item There are two red dots in the graph at random locations. Edit the Mathematica file in the line that looks like
$$\verb!ListPlot[{lorentz[-2.5,5.5,v],lorentz[-3,4.5,v]},PlotStyle ->{PointSize[Large],Red}]!$$
to change the coordinates of those dots so that they mark the space and time coordinates of the closing of the two barn doors in Bob's reference frame. What are those exact coordinates?
\answerspace{0.8in}

\item Describe the sequence of events in Bob's reference frame (move upward in time).  An event description would be something like ``right end of pole passes open left barn door'' (the first event), or ``right end of pole crashes through right barn door.''
\answerspace{1.1in}


\item Now move the slider back to Anna's reference frame. Describe the sequence of events according to Anna.  Hint: The first event is ``left hand door passes over right end of pole''.
\answerspace{1.1in}


\item So, was the pole ever closed inside the barn in either frame? Who has the ``correct'' view —Anna? Bob? Neither? Both?  (The resolution lies in the idea that different observers can measure different times and events that are simultaneous in one frame may not be simultaneous in another.)
\answerspace{1.1in}

\end{enumerate}
