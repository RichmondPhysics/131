
\section{Rolling}

\makelabheader %(Space for student name, etc., defined in master.tex or labmanual_formatting_commands.tex)

\textbf{Objectives }

To apply our understanding of rotation to the case of rolling without skidding
and discover its fundamental properties. 

\textbf{Apparatus}

\begin{itemize}
\item A spool for rolling. 
\item A meter stick and a ruler 
\item A video analysis system (\textit{Tracker}).
\end{itemize}
\textbf{Overview }

We have extensively studied the linear motion of objects and we recently began
the study of rotational motion. When something rolls it is combining both types
of motion in one. In addition, the motion requires a unique application of forces
and it has some surprising characteristics that are often poorly understood.
In this unit we will investigate these unique features.

\textbf{Activity 1: Rolling --- Guesses and Predictions }

To begin our investigation consider the following questions. You might find
it useful to test some of your answers by simply rolling the spool along the
table and observing its motion.

\begin{enumerate}
\item In your own words, what is rolling?
\answerspace{20mm}

\item How is the rotational motion related to the linear motion?
\answerspace{20mm}

\item Make a sketch of the spool rolling and the forces acting on the spool. What
is the effect, if any, of friction? How would the motion change if there was
no friction in the system at all?
\answerspace{30mm}
\end{enumerate}

\pagebreak[2]
\textbf{Activity 2: Investigating Rolling} 

(a) Make a movie of the spool rolling along your laboratory table by following
these steps. 

\begin{enumerate}
\item Turn the camera on and point it along the long axis of your table. The camera
should be around 1 m from where you will roll the spool along the surface of
the table. Place a ruler somewhere in the field of view where it won't interfere
with the motion and parallel to the direction of motion of the spool. This ruler will
be user later to determine the scale. 
\item Practice rolling the spool in front of the camera a few times to make sure the
camera is pointing so there is no change in the vertical position of the center
of the spool as it moves across the camera's field of view. Make a movie of
the spool rolling at a constant speed along the table. See \textbf{Appendix
\ref{tracker}: Video Analysis Using Tracker} for details on making the movie.
\end{enumerate}
(b) Determine the position of the center of the spool during the spools motion.  To do this task follow the instructions in \textbf{Appendix
\ref{tracker}} for creating and calibrating movie data. On each frame click on the center of the spool. 
When you finish tracking the center next track a point on the rim of the spool by repeating the same steps begining with create point mass.  The table and graph on the right will now show mass B.  Once you are finished taking data you can switch between mass A and mass B.

(c) Create a plot of the horizontal position of the center of the spool versus
time. See \textbf{Appendix -\ref{tracker} or -\ref{excel}} for more details.
Fit your data and extract the speed of the center of the spool. Measure the radius of the spool.
\vspace{5mm}

\( v_{\mbox{\small spool}} \) = \hfill{}\( r_{\mbox{\small spool}} \) = \hfill{}
\vspace{5mm}

(d) What is the general expression for the angular displacement of the point
on the rim relative to the center of the spool in terms of the x and y positions
that you recorded above? (think trig)
\vspace{5mm}

\( \theta _{\mbox{\small rim}}= \) 
\vspace{5mm}

(e) We want to graph the angular displacement of the point on the rim relative
to the center as a function of time. To do this:

\begin{enumerate}
\item Display mass B in the table and plot section of Tracker. 

\item Click on the Coordinate System tab along the top of the software.  If locked is checked, uncheck it. Under Coordinate System select reference frame and choose Mass A (this should be your tracking of the center of the spool).

\item Go to the data table. To display the anglar displacement in the data table, click on Table and select \( \theta \) in the Visible Table Columns window. To switch to radians right click on the \( \theta \) column, select Number then Units and choose radians at the bottom of the window.

\item Graph the angular displacement as a function of time.
\end{enumerate}

(f) Fit your graph of the angular displacement versus time and extract the angular
speed. Calculate the ratio of the speed of the center of the spool and the angular
speed. What are the units of this ratio? Is it close in value to any property
of the spool? Is it related to any property of the spool? Why?
\vspace{5mm}

\( \omega _{\mbox{\small rim}}= \)\hfill{} \( v_{\mbox{\small spool}}/
\omega _{\mbox{\small rim}}= \)\hfill{}
\vspace{10mm}

To determine the angular velocity you used the reference frame of the center of the spool.  Change back to a stationary reference frame by going to the Coordinate System tab along the top of the software.  Under Coordinate System select reference frame and choose Default.

(g) Make a plot of the vertical and horizontal components of the position of
the point on the rim as a function of time.  What is the speed of the point on the rim when it is in contact with the
table? (how do you determine velocity from a position-time graph?) Explain.
\vspace{20mm}

(h) How would you define rolling now? What is the role of friction?

