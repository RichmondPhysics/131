\section{Using 32-bit Applications in Windows}
\label{virtual_machine}

A few of the simulation programs we use for labs are 32-bit applications, which means they don't work in the regular 64-bit version of Windows.  To run them, you need to first start a 32-bit version of Windows (a ``Virtual Machine''), and run the programs there.  This appendix shows you how.

\bigskip\textbf{Instructions for Running Applications:}

\begin{quote}
\textbf{Step 1:}
 
Click on the Start menu and access the Physics Applications folder.  Next, click on the link for VM for 32-bit Apps link.  This will open a new window on the desktop.

\medskip\textbf{Step 2:}

In the new window click the L1-GSC-v link.  Next, click the Play Virtual Machine button.  Windows will start up in the new window.   WAIT (for a long time!) for the new window to turn blue.  When it does, ignore instructions on screen. 

\medskip\textbf{Step 3:} 

Instead, click the Send Ctrl-Alt-Del button 
(\includegraphics{appendices/virtual_machine/ctrl_alt_del.eps}) 
located in the upper-left portion of the VM.  Enter your NetID and password to log into the VM.  It will take a while for the start button to appear.

\medskip\textbf{Step 4:}

Click on the Start menu in the VM.  Next, click on the Physics Applications folder in the VM to access the assigned simulation (i.e. Atoms in Motion or EMField)

\end{quote}

\bigskip\textbf{Instructions for Printing From Simulations:}

\begin{quote}
\textbf{Step 1:}
 
Click on desktop of the Primary Machine screen and then click Alt/Print Screen.  

\medskip\textbf{Step 2:}
 
Bring up ``Paint'' from the Accessories folder (Start Menu) on main screen.

\medskip\textbf{Step 3:}
 
Click ``paste'' on the Paint window.

\medskip\textbf{Step 4:}
 
Click ``print'' on the Paint window.
\end{quote}


